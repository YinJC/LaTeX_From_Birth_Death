% \iffalse meta-comment
%
% Copyright (C) 2020-2023
% The LaTeX Project and any individual authors listed elsewhere
% in this file.
%
% This file is part of the LaTeX base system.
% -------------------------------------------
%
% It may be distributed and/or modified under the
% conditions of the LaTeX Project Public License, either version 1.3c
% of this license or (at your option) any later version.
% The latest version of this license is in
%    http://www.latex-project.org/lppl.txt
% and version 1.3c or later is part of all distributions of LaTeX
% version 2008 or later.
%
% This file has the LPPL maintenance status "maintained".
%
% The list of all files belonging to the LaTeX base distribution is
% given in the file `manifest.txt'. See also `legal.txt' for additional
% information.
%
% The list of derived (unpacked) files belonging to the distribution
% and covered by LPPL is defined by the unpacking scripts (with
% extension .ins) which are part of the distribution.
%
% \fi
% Filename: usrguide.tex

\documentclass{ltxguide}

\usepackage[T1]{fontenc}  % needed for \textbackslash in tt
\usepackage{csquotes}
\usepackage{url}

\title{\LaTeX\ for authors --- current version}
\author{\copyright~Copyright 2020-2023, \LaTeX\ Project Team.\\
   All rights reserved.%
   \footnote{This file may be distributed and/or modified under the
     conditions of the \LaTeX{} Project Public License, either version 1.3c
     of this license or (at your option) any later version. See the source
    \texttt{usrguide.tex} for full details.}%
}

\date{2023-05-23}

\NewDocumentCommand\cs{m}{\texttt{\textbackslash\detokenize{#1}}}
\NewDocumentCommand\marg{m}{\arg{#1}}
\NewDocumentCommand\meta{m}{\ensuremath{\langle}\textit{#1}\ensuremath{\rangle}}
\NewDocumentCommand\pkg{m}{\textsf{#1}}
\NewDocumentCommand\text{m}{\ifmmode\mbox{#1}\else#1\fi}
% Fix a 'feature'
\makeatletter
\renewcommand \verbatim@font {\normalfont \ttfamily}
\makeatother


% for fpeval documentation

\providecommand\fpop[1]{\mathop{\texttt{#1}}}
\providecommand\fpbin[1]{\mathbin{\texttt{#1}}}
\providecommand\fprel[1]{\mathrel{\texttt{#1}}}
\providecommand\nan{\texttt{NaN}}

\ExplSyntaxOn
\ProvideExpandableDocumentCommand \fpeval { m } { \fp_eval:n {#1} }
\ExplSyntaxOff

\begin{document}

\maketitle

\tableofcontents

\section{Introduction}

\LaTeXe{} was released in 1994 and added a number of then-new concepts to
\LaTeX{}. These are described in \texttt{usrguide-historic}, which has largely remained
unchanged. Since then, the \LaTeX{} team have worked on a number of ideas,
firstly a programming language for \LaTeX{} (\pkg{expl3}) and then a range of
tools for document authors which build on that language. Here, we describe
\emph{stable} and \emph{widely-usable} concepts that have resulted from that
work. These `new' ideas have been transferred from development packages
into the \LaTeXe{} kernel. As such, they are now available to \emph{all}
\LaTeX{} users and have the \emph{same stability} as any other part of the
kernel. The fact that `behind the scenes' they are built on \pkg{expl3}
is useful for the development team, but is not directly important to users.

\section{Creating document commands and environments}

\subsection{Overview}

Creating document commands and environments using the \LaTeX3 toolset is based
around the idea that a common set of descriptions can be used to cover almost
all argument types used in real documents. Thus parsing is reduced to a simple
description of which arguments a command takes: this description provides the
`glue' between the document syntax and the implementation of the
command.

First, we will describe the argument types, then move on to explain how these
can be used to create both document commands and environments. Various more
specialized features are then described, which allow an even richer application
of a simple interface set up.

The details here are intended to help users create document commands in
general. More technical detail, suitable for \TeX{} programmers, is included
in \texttt{interface3}.

\subsection{Describing argument types}

In order to allow each argument to be defined independently, the parser does
not simply need to know the number of arguments for a function, but also the
nature of each one. This is done by constructing an \emph{argument
specification}, which defines the number of arguments, the type of each
argument and any additional information needed for the parser to read the user
input and properly pass it through to internal functions.

The basic form of the argument specifier is a list of letters, where each
letter defines a type of argument. As will be described below, some of the
types need additional information, such as default values. The argument types
can be divided into two, those which define arguments that are mandatory
(potentially raising an error if not found) and those which define optional
arguments. The mandatory types
\begin{itemize}
  \item[\texttt{m}] A standard mandatory argument, which can either be a single token
    alone or multiple tokens surrounded by curly braces |{}|. Regardless of the
    input, the argument will be passed to the internal code without the outer
    braces. This is the type specifier for a normal \TeX{} argument.
  \item[\texttt{r}] Given as \texttt{r}\meta{token1}\meta{token2}, this denotes a
    `required' delimited argument, where the delimiters are
    \meta{token1} and \meta{token2}. If the opening delimiter \meta{token1} is
    missing, the default marker |-NoValue-| will be inserted after a suitable
    error.
  \item[\texttt{R}] Given as \texttt{R}\meta{token1}\meta{token2}\marg{default},
    this is a `required' delimited argument as for~\texttt{r},
    but it has a user-definable recovery \meta{default} instead of
    |-NoValue-|.
  \item[\texttt{v}] Reads an argument `verbatim', between the following
    character and its next occurrence, in a way similar to the argument
    of the \LaTeXe{} command \cs{verb}. Thus a \texttt{v}-type argument
    is read between two identical characters, which cannot be any of |%|, |\|,
    |#|, |{|, |}| or \verb*| |. The verbatim argument can also be enclosed
    between braces, |{| and |}|. A command with a verbatim argument will
    produce an error when it appears within an argument of another function.
  \item[\texttt{b}] Only suitable in the argument specification of an
    environment, it denotes the body of the environment, between
    |\begin|\marg{environment} and |\end|\marg{environment}.  See
    Section~\ref{sec:cmd:body} for details.
\end{itemize}
The types which define optional arguments are:
\begin{itemize}
  \item[\texttt{o}] A standard \LaTeX{} optional argument, surrounded with square
    brackets, which will supply the special |-NoValue-| marker if not given
    (as described later).
  \item[\texttt{d}] Given as \texttt{d}\meta{token1}\meta{token2}, an optional
    argument which is delimited by \meta{token1} and \meta{token2}. As with
    \texttt{o}, if no value is given the special marker |-NoValue-| is
    returned.
  \item[\texttt{O}] Given as \texttt{O}\marg{default}, is like \texttt{o}, but returns
    \meta{default} if no value is given.
  \item[\texttt{D}] Given as \texttt{D}\meta{token1}\meta{token2}\marg{default}, it is
    as for \texttt{d}, but returns \meta{default} if no value is given.
    Internally, the \texttt{o}, \texttt{d} and \texttt{O} types are
    short-cuts to an appropriated-constructed \texttt{D} type argument.
  \item[\texttt{s}] An optional star, which will result in a value
    \cs{BooleanTrue} if a star is present and \cs{BooleanFalse}
    otherwise (as described later).
  \item[\texttt{t}] An optional \meta{token}, which will result in a value
    \cs{BooleanTrue} if \meta{token} is present and \cs{BooleanFalse}
    otherwise. Given as \texttt{t}\meta{token}.
  \item[\texttt{e}] Given as \texttt{e}\marg{tokens}, a set of optional
    \emph{embellishments}, each of which requires a \emph{value}.
    If an embellishment is not present, |-NoValue-| is returned.  Each
    embellishment gives one argument, ordered as for the list of
    \meta{tokens} in the argument specification.  All \meta{tokens}
    must be distinct.
  \item[\texttt{E}] As for \texttt{e} but returns one or more \meta{defaults}
    if values are not given: \texttt{E}\marg{tokens}\marg{defaults}. See
    Section~\ref{sec:cmd:embellishment} for more details.
\end{itemize}

\subsection{Modifying argument descriptions}

In addition to the argument \emph{types} discussed above, the argument
description also gives special meaning to three other characters.

First, \texttt{+} is used to make an argument long (to accept paragraph
tokens). In contrast to \cs{newcommand}, this applies on an
argument-by-argument basis. So modifying the example to `|s o o +m O{default}|'
means that the mandatory argument is now \cs{long}, whereas the optional
arguments are not.

Secondly, \texttt{!} is used to control whether spaces are allowed before
optional arguments. There are some subtleties to this, as \TeX{} itself
has some restrictions on where spaces can be `detected': more detail
is given in Section~\ref{sec:cmd:opt-space}.

Thirdly, \texttt{=} is used to declare that the following argument should
be interpreted as a series of keyvals. See Section~\ref{sec:cmd:keyval}
for more details.

Finally, the character \texttt{>} is used to declare so-called
`argument processors', which can be used to modify the contents of an
argument before it is passed to the macro definition. The use of argument
processors is a somewhat advanced topic, (or at least a less commonly used
feature) and is covered in Section~\ref{sec:cmd:processors}.

\subsection{Creating document commands and environments}

\begin{decl}
  |\NewDocumentCommand|     \arg{cmd} \arg{arg spec} \arg{code} \\
  |\RenewDocumentCommand|   \arg{cmd} \arg{arg spec} \arg{code} \\
  |\ProvideDocumentCommand| \arg{cmd} \arg{arg spec} \arg{code} \\
  |\DeclareDocumentCommand| \arg{cmd} \arg{arg spec} \arg{code}
\end{decl}
This family of commands are used to create a \meta{cmd}. The argument
specification for the function is given by \meta{arg spec}, and the command
uses the \meta{code} with |#1|, |#2|, etc.\ replaced by the arguments found
by the parser.

An example:
\begin{verbatim}
\NewDocumentCommand\chapter{s o m}
  {%
    \IfBooleanTF{#1}%
      {\typesetstarchapter{#3}}%
      {\typesetnormalchapter{#2}{#3}}%
  }
\end{verbatim}
would be a way to define a \cs{chapter} command which would essentially behave
like the current \LaTeXe{} command (except that it would accept an optional
argument even when a \texttt{*} was parsed). The \cs{typesetnormalchapter}
could test its first argument for being |-NoValue-| to see if an optional
argument was present. (See Section~\ref{sec:cmd:special} for details of
\cs{IfBooleanTF} and testing for |-NoValue-|.)

The difference between the \cs{New...} \cs{Renew...}, \cs{Provide...}
and \cs{Declare...} versions is the behavior if \meta{cmd} is already
defined.
\begin{itemize}
 \item \cs{NewDocumentCommand} will issue an error if \meta{cmd}
   has already been defined.
 \item \cs{RenewDocumentCommand} will issue an error if \meta{cmd}
   has not previously been defined.
 \item \cs{ProvideDocumentCommand} creates a new definition for
   \meta{function} only if one has not already been given.
 \item \cs{DeclareDocumentCommand} will always create the new
   definition, irrespective of any existing \meta{cmd} with the
   same name.  This should be used sparingly.
\end{itemize}

If the \meta{cmd} can't be provided as a single token but needs
\enquote{constructing}, you can use \cs{ExpandArgs} as explained in
Section~\ref{sec:preconstructing-csnames} which also gives an example
in which this is needed.

\begin{decl}
  |\NewDocumentEnvironment|     \arg{env} \arg{arg spec} \arg{beg-code} \arg{end-code} \\
  |\RenewDocumentEnvironment|   \arg{env} \arg{arg spec} \arg{beg-code} \arg{end-code} \\
  |\ProvideDocumentEnvironment| \arg{env} \arg{arg spec} \arg{beg-code} \arg{end-code} \\
  |\DeclareDocumentEnvironment| \arg{env} \arg{arg spec} \arg{beg-code} \arg{end-code}
\end{decl}
These commands work in the same way as \cs{NewDocumentCommand}, etc.\@, but
create environments (\cs{begin}\arg{env} \ldots{}
\cs{end}\arg{env}). Both the \meta{beg-code} and \meta{end-code}
may access the arguments as defined by \meta{arg spec}. The arguments will be
given following \cs{begin}\arg{env}.

\subsection{Optional arguments}
\label{sec:cmd:opt}

In contrast to commands created using \LaTeXe{}'s \cs{newcommand}, optional
arguments created using \cs{NewDocumentCommand} may safely be nested. Thus for
example, following
\begin{verbatim}
\NewDocumentCommand\foo{om}{I grabbed `#1' and `#2'}
\NewDocumentCommand\baz{o}{#1-#1}
\end{verbatim}
using the command as
\begin{verbatim}
\foo[\baz[stuff]]{more stuff}
\end{verbatim}
will print
\begin{quote}
I grabbed `stuff-stuff' and `more stuff'
\end{quote}
This is particularly useful when placing a command with an optional argument
\emph{inside} the optional argument of a second command.

When an optional argument is followed by a mandatory argument with the same
delimiter, the parser issues a warning because the optional argument could not
be omitted by the user, thus becoming in effect mandatory. This can apply to
\texttt{o}, \texttt{d}, \texttt{O}, \texttt{D}, \texttt{s}, \texttt{t},
\texttt{e}, and \texttt{E} type arguments followed by \texttt{r} or
\texttt{R}-type required arguments.

The default for \texttt{O}, \texttt{D} and \texttt{E} arguments can be
the result of grabbing another argument. Thus for example
\begin{verbatim}
\NewDocumentCommand\foo{O{#2} m}
\end{verbatim}
would use the mandatory argument as the default for the leading optional
one.

\subsection{Spacing and optional arguments}
\label{sec:cmd:opt-space}

\TeX{} will find the first argument after a function name irrespective of any
intervening spaces. This is true for both mandatory and optional arguments. So
|\foo[arg]| and \verb*|\foo [arg]| are equivalent. Spaces are also ignored when
collecting arguments up to the last mandatory argument to be collected (as it
must exist). So after
\begin{verbatim}
\NewDocumentCommand\foo{m o m}{ ... }
\end{verbatim}
the user input |\foo{arg1}[arg2]{arg3}| and \verb*|\foo{arg1} [arg2] {arg3}|
will both be parsed in the same way.

The behavior of optional arguments \emph{after} any mandatory arguments is
selectable. The standard settings will allow spaces here, and thus with
\begin{verbatim}
\NewDocumentCommand\foobar{m o}{ ... }
\end{verbatim}
both |\foobar{arg1}[arg2]| and \verb*|\foobar{arg1} [arg2]| will find an
optional argument. This can be changed by giving the modified |!| in the
argument specification:
\begin{verbatim}
\NewDocumentCommand\foobar{m !o}{ ... }
\end{verbatim}
where \verb*|\foobar{arg1} [arg2]| will not find an optional argument.

There is one subtlety here due to the difference in handling by \TeX{} of
`control symbols', where the command name is made up of a single
character, such as `\texttt{\textbackslash\textbackslash}'.
Spaces are not ignored by \TeX{} here,
and thus it is possible to require an optional argument directly follow such a
command. The most common example is the use of \texttt{\textbackslash\textbackslash}
in \pkg{amsmath} environments, which in the terms here would be defined as
\begin{verbatim}
\NewDocumentCommand\\{!s !o}{ ... }
\end{verbatim}

Also notable when using optional arguments in the last position is that \TeX{}
will necessarily look ahead for the argument opening token. This means that
the value of |\inputlineno| will be `out by one' if such a trailing optional
argument is \emph{not} present and the command ends a line; it will be one
greater than the line number containing the last mandatory argument.

\subsection{`Embellishments'}
\label{sec:cmd:embellishment}

The \texttt{E}-type argument allows one default value per test token. This is
achieved by giving a list of defaults for each entry in the list, for example:
\begin{verbatim}
E{^_}{{UP}{DOWN}}
\end{verbatim}
If the list of default values is \emph{shorter} than the list of test tokens,
the special |-NoValue-| marker will be returned (as for the \texttt{e}-type
argument). Thus for example
\begin{verbatim}
E{^_}{{UP}}
\end{verbatim}
has default \texttt{UP} for the |^| test character, but will return the
|-NoValue-| marker as a default for |_|. This allows mixing of explicit
defaults with testing for missing values.


\subsection{Testing special values}
\label{sec:cmd:special}

Optional arguments make use of dedicated variables to return information about
the nature of the argument received.



\begin{decl}
  |\IfNoValueTF| \arg{arg} \arg{true code} \arg{false code} \\
  |\IfNoValueT|  \arg{arg} \arg{true code} \\
  |\IfNoValueF|  \arg{arg} \arg{false code}
\end{decl}
The \cs{IfNoValue(TF)} tests are used to check if \meta{argument} (|#1|,
|#2|, \emph{etc.}) is the special |-NoValue-| marker. For example
\begin{verbatim}
\NewDocumentCommand\foo{o m}
  {%
    \IfNoValueTF {#1}%
      {\DoSomethingJustWithMandatoryArgument{#2}}%
      {\DoSomethingWithBothArguments{#1}{#2}}%
  }
\end{verbatim}
will use a different internal function if the optional argument
is given than if it is not present.

Note that three tests are available, depending on which outcome
branches are required: \cs{IfNoValueTF}, \cs{IfNoValueT} and
\cs{IfNoValueF}.

As the \cs{IfNoValue(TF)} tests are expandable, it is possible to
test these values later, for example at the point of typesetting or
in an expansion context.

It is important to note that |-NoValue-| is constructed such that it
will \emph{not} match the simple text input |-NoValue-|, i.e.~that
\begin{verbatim}
\IfNoValueTF{-NoValue-}
\end{verbatim}
will be logically \texttt{false}.
When two optional arguments follow each other (a syntax we typically
discourage), it can make sense to allow users of the command to
specify only the second argument by providing an empty first
argument.
\NEWdescription{2022/06/01}
Rather than testing separately for emptiness and for
|-NoValue-| it is then best to use the argument type~|O| with an
empty default value, and then test for emptiness using the
conditional \cs{IfBlankTF} (described below) instead.

\begin{decl}
  |\IfValueTF| \arg{arg} \arg{true code} \arg{false code} \\
  |\IfValueT|  \arg{arg} \arg{true code} \\
  |\IfValueF|  \arg{arg} \arg{false code}
\end{decl}
The reverse form of the \cs{IfNoValue(TF)} tests are also available
as \cs{IfValue(TF)}. The context will determine which logical
form makes the most sense for a given code scenario.





\begin{decl}[2022/06/01]
  |\IfBlankTF| \arg{arg} \arg{true code} \arg{false code} \\
  |\IfBlankT|  \arg{arg} \arg{true code} \\
  |\IfBlankF|  \arg{arg} \arg{false code}
\end{decl}


The \cs{IfNoValueTF} command chooses the \meta{true code} if the optional argument has not
been used at all (and it returns the special \texttt{-NoValue-}
marker), but not if it has been given an empty value. In contrast
\cs{IfBlankTF} returns true if its argument is either truly empty or
only contains one or more normal blanks.
For example
\begin{verbatim}
\NewDocumentCommand\foo{m!o}{\par #1:
  \IfNoValueTF{#2}
    {No optional}%
    {%
      \IfBlankTF{#2}
        {Blanks in or empty}%
        {Real content in}%
    }%
  \space argument!}
\foo{1}[bar] \foo{2}[  ] \foo{3}[] \foo{4}[\space] \foo{5} [x]
\end{verbatim}
results
in the following output:
\begin{quote}
  \NewDocumentCommand\foo{m!o}{\par #1:
    \IfNoValueTF{#2}{No optional}%
                {\IfBlankTF{#2}{Blanks in or empty}%
                  {Real content in}}%
                \space argument!}
  \foo{1}[bar] \foo{2}[  ] \foo{3}[] \foo{4}[\space] \foo{5} [x]
\end{quote}

Note that the \cs{space} in (4) is considered real content---because it is
a command and not a \enquote{space} character---even though it results
in producing a space. You can also observe in (5) the effect of the
\texttt{!} specifier, preventing the last \cs{foo}
from interpreting \texttt{[x]} as its optional argument.




\begin{decl}
  |\BooleanFalse|  \\
  |\BooleanTrue|
\end{decl}
The \texttt{true} and \texttt{false} flags set when searching for
an optional character (using \texttt{s} or \texttt{t\meta{char}}) have
names which are accessible outside of code blocks.

\begin{decl}
  |\IfBooleanTF| \arg{arg} \arg{true code} \arg{false code} \\
  |\IfBooleanT|  \arg{arg} \arg{true code} \\
  |\IfBooleanF|  \arg{arg} \arg{false code}
\end{decl}
Used to test if \meta{argument} (|#1|, |#2|, \emph{etc.}) is
\cs{BooleanTrue} or \cs{BooleanFalse}. For example
\begin{verbatim}
\NewDocumentCommand\foo{sm}
  {%
    \IfBooleanTF {#1}%
      {\DoSomethingWithStar{#2}}%
      {\DoSomethingWithoutStar{#2}}%
  }
\end{verbatim}
checks for a star as the first argument, then chooses the action to
take based on this information.

\subsection{Auto-converting to key--value format}
\label{sec:cmd:keyval}

Some document commands have a long history of accepting a `free text' optional
argument, for example \cs{caption} and the sectioning commands \cs{section},
etc. Introducing more sophisticated (keyval) options to these commands
therefore needs a method to interpret the optional argument \emph{either} as
free text \emph{or} as a series of keyvals. This needs to take place
during argument grabbing as there is a need for careful treatment of
braces to obtain the correct result.

The \texttt{=} modifier is available to allow \pkg{ltcmd} to correctly
implement this process. The modifier guarantees that the argument will be
passed to further code as a series of keyvals. To do that, the \texttt{=}
should be followed by an argument containing the default key name. This is used
as the key in a key--value pair \emph{if} the `raw' argument does \emph{not}
have the correct form to be interpreted as a set of keyvals.

Taking \cs{caption} as an example, with the demonstration implementation
\begin{verbatim}
\DeclareDocumentCommand
  \caption
  {s ={short-text} +O{#3} +m}
  {%
    \showtokens{Grabbed arguments:^^J(#2)^^Jand^^J(#3)}%
  }
\end{verbatim}
the default key name is \texttt{short-text}. When the command \cs{caption} is
then used, if the optional argument is free text such as
\begin{verbatim}
\caption[Some short text]{A much longer and more detailed text for
  demonstration purposes}
\end{verbatim}
then the output will be
\begin{verbatim}
Grabbed arguments:
(short-text={Some short text})
and
(A much longer and more detailed text for demonstration purposes)
\end{verbatim}
On the other hand, if the caption is given with a keyval-form argument
\begin{verbatim}
\caption[label = cap:demo]%
  {A much longer and more detailed text for demonstration purposes}
\end{verbatim}
then this will be respected
\begin{verbatim}
Grabbed arguments:
(label = cap:demo)
and
(A much longer and more detailed text for demonstration purposes)
\end{verbatim}

Interpretation as keyval form is determined by the presence of \texttt{=}
characters within the argument. Those in inline math mode (enclosed within
\verb|$...$| or \verb|\(...\)|) are ignored. An argument can be forced to be
read as keyvals by including an empty entry at the start
\begin{verbatim}
\caption[=,This is now a keyval]%
% ...
\caption[This is not $=$ keyval]%
\end{verbatim}

This empty entry is \emph{not} passed to the underlying code, so will not lead
to issues with keyval parsers that do not allow an empty key name. Any text-mode
\texttt{=} signs will need to be braced to avoid being misinterpreted: this
is likely most conveniently handled by bracing the entire argument
\begin{verbatim}
\caption[{Not = to a keyval!}]%
\end{verbatim}
which will be passed correctly as
\begin{verbatim}
Grabbed arguments:
(short-text = {Not = to a keyval!})
\end{verbatim}

\subsection{Argument processors}
\label{sec:cmd:processors}

Argument processor are applied to an argument \emph{after} it has been grabbed
by the underlying system but before it is passed to \meta{code}. An argument
processor can therefore be used to regularize input at an early stage, allowing
the internal functions to be completely independent of input form. Processors
are applied to user input and to default values for optional arguments, but
\emph{not} to the special |-NoValue-| marker.

Each argument processor is specified by the syntax \texttt{>}\marg{processor}
in the argument specification. Processors are applied from right to left, so
that
\begin{verbatim}
>{\ProcessorB} >{\ProcessorA} m
\end{verbatim}
would apply \cs{ProcessorA} followed by \cs{ProcessorB} to the tokens grabbed
by the \texttt{m} argument.

\begin{decl}
  |\SplitArgument| \arg{number} \arg{token(s)}
\end{decl}
This processor splits the argument given at each occurrence of the
\meta{tokens} up to a maximum of \meta{number} tokens (thus
dividing the input into $\text{\meta{number}} + 1$ parts).
An error is given if too many \meta{tokens} are present in the
input. The processed input is placed inside
$\text{\meta{number}} + 1$ sets of braces for further use.
If there are fewer than \arg{number} of \arg{tokens} in the argument
then |-NoValue-| markers are added at the end of the processed
argument.
\begin{verbatim}
\NewDocumentCommand \foo {>{\SplitArgument{2}{;}} m}
  {\InternalFunctionOfThreeArguments#1}
\end{verbatim}
If only a single character \meta{token} is used for the split, any
category code $13$ (active) character matching the \meta{token} will
be replaced before the split takes place.
Spaces are trimmed at each end of each item parsed.

The \texttt{E} argument type is somewhat special, because with a single
\texttt{E} in the command declaration you may end up with several
arguments in a command (one formal argument per embellishment token).
Therefore, when an argument processor is applied to an \texttt{E}-type
argument, all the arguments pass through that processor before being fed
to the \meta{code}.  For example, this command
\begin{verbatim}
\NewDocumentCommand \foo { >{\TrimSpaces} e{_^} }
  { [#1](#2) }
\end{verbatim}
applies \cs{TrimSpaces} to both arguments.

\begin{decl}
  |\SplitList| \arg{token(s)}
\end{decl}
This processor splits the argument given at each occurrence of the
\meta{token(s)} where the number of items is not fixed. Each item is
then wrapped in braces within |#1|. The result is that the
processed argument can be further processed using a mapping function
(see below).
\begin{verbatim}
\NewDocumentCommand \foo {>{\SplitList{;}} m}
  {\MappingFunction#1}
\end{verbatim}
If only a single character \meta{token} is used for the split, any
category code $13$ (active) character matching the \meta{token} will
be replaced before the split takes place.
Spaces are trimmed at each end of each item parsed. Exactly one set
of braces will be stripped if an entire item is surrounded by them,
i.e.~the following inputs and outputs result (each separte item as
a brace group).
\begin{verbatim}
a      ==> {a}
{a}    ==> {a}
{a}b   ==> {{a}b}
a,b    ==> {a}{b}
{a},b  ==> {a}{b}
a,{b}  ==> {a}{b}
a,{b}c ==> {a}{{b}c}
\end{verbatim}

\begin{decl}
  |\ProcessList| \arg{list} \arg{function}
\end{decl}
To support \cs{SplitList}, the function \cs{ProcessList} is available
to apply a \meta{function} to every entry in a \meta{list}. The
\meta{function} should absorb one argument: the list entry. For example
\begin{verbatim}
\NewDocumentCommand \foo {>{\SplitList{;}} m}
  {\ProcessList{#1}{\SomeDocumentCommand}}
\end{verbatim}

\begin{decl}
  |\ReverseBoolean|
\end{decl}
This processor reverses the logic of \cs{BooleanTrue} and
\cs{BooleanFalse}, so that the example from earlier would become
\begin{verbatim}
\NewDocumentCommand\foo{>{\ReverseBoolean} s m}
  {%
    \IfBooleanTF#1%
      {\DoSomethingWithoutStar{#2}}%
      {\DoSomethingWithStar{#2}}%
  }
\end{verbatim}

\begin{decl}
  |\TrimSpaces|
\end{decl}
Removes any leading and trailing spaces (tokens with character code~$32$
and category code~$10$) for the ends of the argument. Thus for example
declaring a function
\begin{verbatim}
\NewDocumentCommand\foo {>{\TrimSpaces} m}
  {\showtokens{#1}}
\end{verbatim}
and using it in a document as
\begin{flushleft}
  \verb=  =\verb*=\foo{ hello world }=
\end{flushleft}
will show `\verb*=hello world=' at the terminal, with the space at each
end removed. \cs{TrimSpaces} will remove multiple spaces from the ends of
the input in cases where these have been included such that the standard
\TeX{} conversion of multiple spaces to a single space does not apply.

\subsection{Body of an environment}
\label{sec:cmd:body}

While environments |\begin|\marg{environment}\ \dots{}\,|\end|\marg{environment}
are typically used in cases where the code implementing the \meta{environment}
does not need to access the contents of the environment (its `body'),
it is sometimes useful to have the body as a standard argument.

This is achieved by ending the argument specification with~\texttt{b}, which is
a dedicated argument type for this situation. For instance
\begin{verbatim}
\NewDocumentEnvironment{twice} {O{\ttfamily} +b}
  {#2#1#2} {}
\begin{twice}[\itshape]
  Hello world!
\end{twice}
\end{verbatim}
typesets `Hello world!{\itshape Hello world!}'.

The prefix |+| is used to allow multiple paragraphs in the environment's body.
Argument processors can also be applied to \texttt{b}~arguments. By default,
spaces are trimmed at both ends of the body: in the example there would
otherwise be spaces coming from the ends the lines after |[\itshape]| and
|world!|. Putting the prefix |!| before \texttt{b} suppresses space-trimming.

When \texttt{b} is used in the argument specification, the last
argument of the environment declaration (e.g.,
\cs{NewDocumentEnvironment}), which consists of an \meta{end code} to
insert at |\end|\marg{environment}, is redundant since one can simply
put that code at the end of the \meta{start code}. Nevertheless this
(empty) \meta{end code} must be provided.

Environments that use this feature can be nested.

\subsection{Fully-expandable document commands\label{sec:ltcmd:expandable}}

Document commands created using \cs{NewDocumentCommand}, etc.\@, are normally
created so that they do not expand unexpectedly. This is done using engine
features, so is more powerful than \LaTeXe{}'s \cs{protect} mechanism. There
are \emph{very rare} occasion when it may be useful to create functions using a
expansion-only grabber. This imposes a number of restrictions on the
nature of the arguments accepted by a function, and the code it implements.
This facility should only be used when \emph{necessary}.

\begin{decl}
  |\NewExpandableDocumentCommand|     \arg{cmd} \arg{arg spec} \arg{code} \\
  |\RenewExpandableDocumentCommand|   \arg{cmd} \arg{arg spec} \arg{code} \\
  |\ProvideExpandableDocumentCommand| \arg{cmd} \arg{arg spec} \arg{code} \\
  |\DeclareExpandableDocumentCommand| \arg{cmd} \arg{arg spec} \arg{code}
\end{decl}
This family of commands is used to create a document-level \meta{function},
which will grab its arguments in a fully-expandable manner. The
argument specification for the function is given by \meta{arg spec},
and the function will execute \meta{code}. In  general, \meta{code} will
also be fully expandable, although it is possible that this will
not be the case (for example, a function for use in a table might
expand so that \cs{omit} is the first non-expandable non-space token).

Parsing arguments by pure expansion imposes a number of restrictions on
both the type of arguments that can be read and the error checking
available:
\begin{itemize}
  \item The last argument (if any are present) must be one of the
    mandatory types \texttt{m}, \texttt{r} or \texttt{R}.
  \item The `verbatim' argument type \texttt{v} is not available.
  \item Argument processors (using \texttt{>}) are not available.
  \item It is not possible to differentiate between, for example
    |\foo[| and |\foo{[}|: in both cases the \texttt{[} will be
    interpreted as the start of an optional argument. As a
    result, checking for optional arguments is less robust than
    in the standard version.
\end{itemize}

\subsection{Commands at the start of tabular cells}

Creating commands that are used at the start of tabular cells imposes
some restrictions on the underlying implementation. The standard \LaTeX{}
tabular environments (\texttt{tabular}, etc.)  use a mechanism which requires
that any command wrapping \cs{multicolumn} or similar must be
`expandable'. This is \emph{not} the case for commands created using
\cs{NewDocumentCommand}, etc., which as detailed in
Section~\ref{sec:ltcmd:expandable} use an engine feature which prevents
such `expansion'. Therefore, to create such wrappers for use at the start
of tabular cells, you must use \cs{NewExpandableDocumentCommand}, for example
\begin{verbatim}
\NewExpandableDocumentCommand\MyMultiCol{m}{\multicolumn{3}{c}{#1}}
\begin{tabular}{lcr}
a & b & c \\
\MyMultiCol{stuff} \\
\end{tabular}
\end{verbatim}

\subsection{Details about argument delimiters}

In normal (non-expandable) commands, the delimited types look for the
initial delimiter by peeking ahead (using \pkg{expl3}'s |\peek_...|
functions) looking for the delimiter token.  The token has to have the
same meaning and `shape' of the token defined as delimiter.
There are three possible cases of delimiters: character tokens, control
sequence tokens, and active character tokens.  For all practical purposes
of this description, active character tokens will behave exactly as
control sequence tokens.

\subsubsection{Character tokens}

A character token is characterized by its character code, and its meaning
is the category code~(|\catcode|).  When a command is defined, the meaning
of the character token is fixed into the definition of the command and
cannot change.  A command will correctly see an argument delimiter if
the open delimiter has the same character and category codes as at the
time of the definition.  For example in:
\begin{verbatim}
\NewDocumentCommand { \foobar } { D<>{default} } {(#1)}
\foobar <hello> \par
\char_set_catcode_letter:N <
\foobar <hello>
\end{verbatim}
the output would be:
\begin{verbatim}
(hello)
(default)<hello>
\end{verbatim}
as the open-delimiter |<| changed in meaning between the two calls to
|\foobar|, so the second one doesn't see the |<| as a valid delimiter.
Commands assume that if a valid open-delimiter was found, a matching
close-delimiter will also be there.  If it is not (either by being
omitted or by changing in meaning), a low-level \TeX{} error is raised
and the command call is aborted.

\subsubsection{Control sequence tokens}

A control sequence (or control character) token is characterized by is
its name, and its meaning is its definition.
A token cannot have two different meanings at the same time.
When a control sequence is defined as delimiter in a command,
it will be detected as delimiter whenever the control sequence name
is found in the document regardless of its current definition.
For example in:
\begin{verbatim}
\cs_set:Npn \x { abc }
\NewDocumentCommand { \foobar } { D\x\y{default} } {(#1)}
\foobar \x hello\y \par
\cs_set:Npn \x { def }
\foobar \x hello\y
\end{verbatim}
the output would be:
\begin{verbatim}
(hello)
(hello)
\end{verbatim}
with both calls to the command seeing the delimiter |\x|.

\subsection{Creating new argument processors}

\begin{decl}
  |\ProcessedArgument|
\end{decl}
Argument processors allow manipulation of a grabbed argument before it is
passed to the underlying code. New processor implementations may be created
as functions which take one trailing argument, and which leave their result in
the \cs{ProcessedArgument} variable. For example, \cs{ReverseBoolean} is
defined as
\begin{verbatim}
\ExplSyntaxOn
\cs_new_protected:Npn \ReverseBoolean #1
  {
    \bool_if:NTF #1
      { \tl_set:Nn \ProcessedArgument { \c_false_bool } }
      { \tl_set:Nn \ProcessedArgument { \c_true_bool } }
  }
\ExplSyntaxOff
\end{verbatim}
[As an aside: the code is written in \pkg{expl3}, so we don't have to
  worry about spaces creeping into the definition.]

\subsection{Access to the argument specification}

The argument specifications for document commands and environments are
available for examination and use.

\begin{decl}
  |\GetDocumentCommandArgSpec| \arg{cmd}         \\
  |\GetDocumentEnvironmentArgSpec| \arg{environment}
\end{decl}
These functions transfer the current argument specification for the
requested \meta{cmd} or \meta{environment} into the token list
variable \cs{ArgumentSpecification}. If the \meta{cmd} or
\meta{environment} has no known argument specification then an error
is issued. The assignment to \cs{ArgumentSpecification} is local to
the current \TeX{} group.

\begin{decl}
  |\ShowDocumentCommandArgSpec| \arg{cmd}         \\
  |\ShowDocumentEnvironmentArgSpec| \arg{environment}
\end{decl}
These functions show the current argument specification for the
requested \meta{cmd} or \meta{environment} at the terminal. If
the \meta{cmd} or \meta{environment} has no known argument
specification then an error is issued.



\section{Copying and showing (robust) commands and environments}

If you want to (slightly) alter an existing command you may want to
save the current definition under a new name and then use that in a
new definition. If the existing command is robust, then the old trick of
using the low-level \cs{let} for this doesn't work, because it only
copies the top-level definition, but not the part that actually does
the work. As most \LaTeX{} commands are nowadays robust, \LaTeX{}
now offers some high-level declarations for this instead.

However, please note that it is usually better to make use of
available hooks (e.g., the generic command or environment hooks),
instead of copying the current definition and thereby freezing it; see
the hook management documentation \texttt{lthooks-doc.pdf} for
details.

\begin{decl}
  |\NewCommandCopy|       \arg{cmd} \arg{existing-cmd} \\
  |\RenewCommandCopy|     \arg{cmd} \arg{existing-cmd} \\
  |\DeclareCommandCopy|   \arg{cmd} \arg{existing-cmd}
\end{decl}

This copies the definition of \meta{existing-cmd} to \meta{cmd}. After
this \meta{existing-cmd} can be redefined and \meta{cmd}
still works! This allows you to then provide a new definition for
\meta{existing-cmd} that makes use of \meta{cmd} (i.e., of its old
definition). For example, after
\begin{verbatim}
\NewCommandCopy\LaTeXorig\LaTeX
\RenewDocumentCommand\LaTeX{}{\textcolor{blue}{\LaTeXorig}}
\end{verbatim}
all \LaTeX{} logos generated with \cs{LaTeX} will come out in blue
(assuming you have a color package loaded).

The differences between \cs{New...} and \cs{Renew...} are as
elsewhere: i.e., you get an error depending on whether or not
\meta{cmd} already exists, or in case of \cs{Declare...} it is copied
regardless. Note that there is no \cs{Provide...} declaration, because
that would be of limited value.


If the \meta{cmd} or \meta{existing-cmd} can't be provided as a single
token but need \enquote{constructing}, you can use \cs{ExpandArgs}
as explained in
Section~\ref{sec:preconstructing-csnames}.

\begin{decl}
  |\ShowCommand|       \arg{cmd}
\end{decl}

This displays the meaning of the \meta{cmd} on the terminal and then
stops (just like the primitive \cs{show}). The difference is that it
correctly shows the meaning of more complex commands, e.g., in case of
robust commands it displays not only the top-level definition but
also the actual payload code and in case of commands declared with
\cs{NewDocumentCommand}, etc.\ it also gives you detailed information
about the argument signature.

\begin{decl}
  |\NewEnvironmentCopy|       \arg{env} \arg{existing-env} \\
  |\RenewEnvironmentCopy|     \arg{env} \arg{existing-env} \\
  |\DeclareEnvironmentCopy|   \arg{env} \arg{existing-env}
\end{decl}

This copies the definition for environment \meta{existing-env} to
\meta{env} (both the beginning and end code), i.e., it is simply
applying \cs{NewCommandCopy} twice to the internal commands that
define an environment, i.e., \cs{}\meta{env} and \cs{end}\meta{env}.
The differences between \cs{New...}, \cs{Renew...}, and
\cs{Declare...} are the usual ones.

\begin{decl}
  |\ShowEnvironment|       \arg{env}
\end{decl}
This displays the meaning of the begin end end code for environment \meta{env}.




\section[Preconstructing command names \\ (or otherwise expanding arguments)]
        {Preconstructing command names (or otherwise expanding arguments)}
\label{sec:preconstructing-csnames}

When declaring new commands with \cs{NewDocumentCommand} or
\cs{NewCommandCopy} or similar, it is sometimes necessary to
``construct'' the csname. As a general mechanism the L3 programming
layer has \cs{exp_args:N...} for this, but there is no mechanism for
it if \cs{ExplSyntaxOn} is not active (and mixing programming and user
interface level commands is not a good approach anyhow). We therefore
offer a mechanism to access this ability using CamelCase naming.

\begin{decl}
  |\UseName|  \arg{string} \\
  |\ExpandArgs| \arg{spec} \arg{cmd} \arg{arg1} \dots
\end{decl}

\cs{UseName} turns the \meta{string} directly into a csname and
then executes it: this is equivalent to the long-standing
\LaTeXe{} internal command \cs{@nameuse}, or the L3 programming
equivalent \cs{use:c}. \cs{ExpandArgs} takes a \meta{spec} which
describes how to expand the \meta{arguments}, carries out these
operations then executes the \meta{cmd}. The \meta{spec} uses
the descriptions offered by the L3 programming layer, and the
relevant \cs{exp_args:N...} function must exist. Common cases will
have a \meta{spec} of \texttt{c}, \texttt{cc} or \texttt{Nc}: see below.

As an example, the following declaration provides a method to generate
copyedit commands:
\begin{verbatim}
\NewDocumentCommand\newcopyedit{mO{red}}
  {%
    \newcounter{todo#1}%
    \ExpandArgs{c}\NewDocumentCommand{#1}{s m}%
      {%
        \stepcounter{todo#1}%
        \IfBooleanTF {##1}%
          {\todo[color=#2!10]{\UseName{thetodo#1}: ##2}}%
          {\todo[inline,color=#2!10]{\UseName{thetodo#1}: ##2}}%
      }%
  }
\end{verbatim}
Given that declaration you can then write
\verb/\newcopyedit{note}[blue]/ which defines the command \cs{note}
and the corresponding counter for you.

A second example is to copy a command by string name using
\cs{NewCommandCopy}: here we might need to construct both command
names.
\begin{verbatim}
\NewDocumentCommand\savebyname{m}
  {\ExpandArgs{cc}\NewCommandCopy{saved#1}{#1}}
\end{verbatim}

In the \meta{spec} each \texttt{c} stands for one argument that is
turned into a `\texttt{c}'ommand. An \texttt{n} represents a
`\texttt{n}'ormal argument that is not altered and \texttt{N} stands for
a `\texttt{N}'ormal argument which is also left unchanged, but one
consisting only of a single token (and usually unbraced). Thus, to
construct a command from a string only for the second argument of
\cs{NewCommandCopy} you would write
\begin{verbatim}
\ExpandArgs{Nc}\NewCommandCopy\mysectionctr{c@section}
\end{verbatim}
There are several other single letters supported in the L3 programming
layer that \emph{could} be used in the \meta{spec} to manipulate
arguments in other ways.  If you are interested, take a look at the
\enquote{Argument expansion} section in the L3 programming layer
documentation in \texttt{interface3.pdf}.



\section{Expandable floating point (and other) calculations}

The \LaTeX3 programming layer which is part of the format offers a
rich interface to manipulate floating point variables and values. To
allow for (simpler) applications to use this on document-level or in
packages otherwise not making use of the L3 programming layer a few
interface commands are made available.


\begin{decl}
  |\fpeval| \arg{floating point expression}
\end{decl}

The expandable command \cs{fpeval} takes as its argument a floating
point expression and produces a result using the normal rules of
mathematics. As this command is expandable it can be used where \TeX{}
requires a number and for example within a low-level \cs{edef} operation
to give a purely numerical result.



Briefly, the floating point expressions may comprise:
\begin{itemize}
  \item Basic arithmetic: addition $x\fpbin{+}y$, subtraction $x\fpbin{-}y$,
    multiplication $x\fpbin{*}y$, division $x\fpbin{/}y$, square root~$\sqrt{x}$,
    and parentheses.
  \item Comparison operators: $x\fprel{<}y$,
    $x\fprel{<=}y$, $x\fprel{>?}y$,
    $x\fprel{!=}y$ \emph{etc.}
  \item Boolean logic: sign $\fpop{sign} x$,
    negation $\fpop{!}x$, conjunction
    $x\fprel{\&\&}y$, disjunction $x\fprel{\string|\string|}y$, ternary
    operator $x\fprel{?}y\fprel{:}z$.
  \item Exponentials: $\fpop{exp} x$, $\fpop{ln} x$, $x\mathord{\texttt{\^{}}}y$.
  \item Integer factorial: $\fpop{fact} x$.
  \item Trigonometry: $\fpop{sin} x$, $\fpop{cos} x$, $\fpop{tan} x$, $\fpop{cot} x$, $\fpop{sec}
    x$, $\fpop{csc} x$ expecting their arguments in radians, and
    $\fpop{sind} x$, $\fpop{cosd} x$,
    $\fpop{tand} x$, $\fpop{cotd} x$,
    $\fpop{secd} x$, $\fpop{cscd} x$ expecting their
    arguments in degrees.
  \item Inverse trigonometric functions: $\fpop{asin} x$,
    $\fpop{acos} x$, $\fpop{atan} x$,
    $\fpop{acot} x$, $\fpop{asec} x$,
    $\fpop{acsc} x$ giving a result in radians, and
    $\fpop{asind} x$, $\fpop{acosd} x$,
    $\fpop{atand} x$, $\fpop{acotd} x$,
    $\fpop{asecd} x$, $\fpop{acscd} x$ giving a result
    in degrees.
  \item Extrema: $\fpop{max}(x_{1},x_{2},\ldots)$, $\fpop{min}(x_{1},x_{2},\ldots)$,
    $\fpop{abs}(x)$.
  \item Rounding functions, controlled by two optional
    values,  $n$ (number of places, $0$ by default) and
      $t$ (behavior on a tie, $\nan$ by default):
    \begin{itemize}
    \item $\fpop{trunc}(x,n)$ rounds towards zero,
    \item $\fpop{floor}(x,n)$ rounds towards~$-\infty$,
    \item $\fpop{ceil}(x,n)$ rounds towards~$+\infty$,
    \item $\fpop{round}(x,n,t)$ rounds to the closest value, with
    ties rounded to an even value by default, towards zero if $t=0$,
    towards $+\infty$ if $t>0$ and towards $-\infty$ if $t<0$.
    \end{itemize}
  \item Random numbers: $\fpop{rand}()$, $\fpop{randint}(m,n)$.
  \item Constants: \texttt{pi}, \texttt{deg} (one degree in radians).
  \item Dimensions, automatically expressed in points, \emph{e.g.},
    \texttt{pc} is~$12$.
  \item Automatic conversion (no need for \cs{number}) of
    integer, dimension, and skip variables to floating points numbers,
    expressing dimensions in points and ignoring the stretch and
    shrink components of skips.
  \item Tuples: $(x_1,\ldots{},x_n)$ that can be added together,
    multiplied or divided by a floating point number, and nested.
\end{itemize}

An example of use could be the following:
\begin{verbatim}
\LaTeX{} can now compute: $ \frac{\sin (3.5)}{2} + 2\cdot 10^{-3}
  = \fpeval{sin(3.5)/2 + 2e-3} $.
\end{verbatim}
which produces the following output:
\begin{quote}
  \LaTeX{} can now compute: $ \frac{\sin (3.5)}{2} + 2\cdot 10^{-3}
  = \fpeval{sin(3.5)/2 + 2e-3} $.
\end{quote}

\begin{decl}
  |\inteval| \arg{integer expression}
\end{decl}

  The expandable command \cs{inteval} takes as its argument an integer
  expression and produces a result using the normal rules of
  mathematics with some restrictions, see below. The operations
  recognized are |+|, |-|, |*| and |/| plus parentheses.  As this
  command is expandable it can be used where \TeX{} requires a number
  and for example within a low-level \cs{edef} operation to give a
  purely numerical result.

  This is basically a thin wrapper for the primitive \cs{numexpr}
  command and therefore has some syntax restrictions. These are:
  \begin{itemize}
  \item \texttt{/} denotes division rounded to the closest integer with
    ties rounded away from zero;
  \item there is an error and the overall expression evaluates to zero
    whenever the absolute value of any intermediate result exceeds
    $2^{31}-1$, except in the case of scaling operations
    $a$\texttt{*}$b$\texttt{/}$c$, for which $a$\texttt{*}$b$ may be
    arbitrarily large;
  \item parentheses may not appear after unary \texttt{+} or
    \texttt{-}, namely placing \texttt{+(} or \texttt{-(} at the start
    of an expression or after \texttt{+}, \texttt{-}, \texttt{*},
    \texttt{/} or~\texttt{(} leads to an error.
  \end{itemize}

An example of use could be the following.
\begin{verbatim}
\LaTeX{} can now compute: The sum of the numbers is $\inteval{1 + 2 + 3}$.
\end{verbatim}
which results in
\enquote{\LaTeX{} can now compute: The sum of the numbers is $\inteval{1 + 2 + 3}$.}


\begin{decl}
  |\dimeval| \arg{dimen expression} \qquad
  |\skipeval| \arg{skip expression}
\end{decl}

Similar to \cs{inteval} but computing a length (\texttt{dimen}) or a
rubber length (\texttt{skip}) value. Both are thin wrappers around
the corresponding engine primitives, which makes them fast, but
therefore shows the same syntax peculiars as discussed
above. Nevertheless, in practice they are usually sufficient.  For
example
\begin{verbatim}
\newcommand\calculateheight[1]{%
  \setlength\textheight{\dimeval{\topskip+\baselineskip*\inteval{#1-1}}}}
\end{verbatim}
sets the \cs{textheight} to the appropriate value if a page should
hold a specific number of text lines. Thus after |\calculateheight{40}|
it is set to \dimeval{\topskip+\baselineskip*\inteval{40-1}}, given
the values \cs{topskip} (\dimeval{\topskip}) and \cs{baselineskip}
(\dimeval{\baselineskip}) in the current document.

\section{Case changing}

\TeX{} provides two primitives \cs{uppercase} and \cs{lowercase} for changing
the case of text. However, these have a range of limitations: they only change
the case of explicit characters, do not account for the surrounding context, do
not support UTF-8 input with 8-bit engines, etc. To overcome this problem,
\LaTeX{} provides the commands \cs{MakeUppercase}, \cs{MakeLowercase} and
\cs{MakeTitlecase}: these offer significant enhancement over the \TeX{}
primitives. These commands are engine-robust (\cs{protected}), and so
can be used in moving arguments.

Upper- and lower-casing are well-understood in general conversation.
Titlecasing here follows the definition given by the Unicode Consortium: the
first character of the input will be converted to (broadly) uppercase, and the
rest of the input to lowercase. The full range of Unicode UTF-8 input can be
supported, with the proviso that at present the characters set up with 8-bit
engines match those available in standard input encodings (|T1|, |T2|, |LGR|).
\begin{quotation}
  \begin{tabular}{rl}
    |\MakeUppercase{hello WORLD ßüé}| & \MakeUppercase{hello WORLD ßüé} \\
    |\MakeLowercase{hello WORLD ßüé}| & \MakeLowercase{hello WORLD ßüé} \\
    |\MakeTitlecase{hello WORLD ßüé}| & \MakeTitlecase{hello WORLD ßüé} \\
  \end{tabular}
\end{quotation}

The case-changing commands take an optional argument which can be used to
tailor the output. This optional argument accepts the key \texttt{locale},
also available under the alias \texttt{lang},
which can be used to give a language identifier in BCP-47
format. This is then applied to select language-specific features during
case-changing.

The input given to these commands is `expanded' before case changing is
applied. This means that any commands within the input that convert to pure
text will be case changed. Mathematical content is automatically excluded, as
are the arguments to the commands \cs{label}, \cs{ref}, \cs{cite}, \cs{begin}
and \cs{end}. Additional exclusions can be added using the command
\cs{AddToNoCaseChangeList}. Input can be excluded from case changing using the
command \cs{NoCaseChange}.
\begin{quotation}
  \begin{tabular}{rl}
    |\MakeUppercase{Some text $y = mx + c$}|
      & \MakeUppercase{Some text $y = mx + c$} \\
    |\MakeUppercase{\NoCaseChange{iPhone}}|
      & \MakeLowercase{\NoCaseChange{iPhone}} \\
  \end{tabular}
\end{quotation}

To allow robust commands to be used within case changing \emph{and} to produce
the expected output, two additional control commands are available.
\cs{CaseSwitch} allows the user to specify the result for the four possible
cases
\begin{itemize}
  \item No case changing
  \item Uppercasing
  \item Lowercasing
  \item Titlecasing (only applies for the start of the input)
\end{itemize}

The command \cs{DeclareCaseChangeEquivalent} provides a way to substitute a
command by an alternative version when it is found inside a case changing
situation. There are three commands for customising the case changing of
codepoints
\begin{decl}
  |\DeclareLowercaseMapping| \oarg{locale} \arg{codepoint} \arg{output} \\
  |\DeclareTitlecaseMapping| \oarg{locale} \arg{codepoint} \arg{output} \\
  |\DeclareUppercaseMapping| \oarg{locale} \arg{codepoint} \arg{output}
\end{decl}
All three take a \meta{codepoint} (as an integer expression) and will
result in the \meta{output} being produced under the appropriate case changing
operation. The optional \meta{locale} can be given if the mapping should only
apply to a specific one: this is given in BCP-47 format
(\url{https://en.wikipedia.org/wiki/IETF_language_tag}). For example,
the kernel customises the mapping for U+01F0 (\v{j}) when uppercasing in
8-bit engines:
\begin{quotation}
  |\DeclareUppercaseMapping{"01F0}{\v{J}}|
\end{quotation}
as there is no pre-composed \v{J} character, and this is problematic if
the engine does not support Unicode natively. Similarly, to set a locale
\texttt{xx} to behave in the same way as Turkish and retain the difference
between dotted- and dotless-i, one could use for example
\begin{quotation}
  |\DeclareLowercaseMapping[xx]{"0049}{\i}|\\
  |\DeclareLowercaseMapping[xx]{"0130}{i}|\\
  |\DeclareUppercaseMapping[xx]{"0069}{\.{I}}|\\
  |\DeclareUppercaseMapping[xx]{"0131}{I}|
\end{quotation}

\end{document}
