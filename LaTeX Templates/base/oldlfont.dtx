% \iffalse meta-comment
%
% Copyright (C) 1993-2023
% The LaTeX Project and any individual authors listed elsewhere
% in this file.
%
% This file is part of the LaTeX base system.
% -------------------------------------------
%
% It may be distributed and/or modified under the
% conditions of the LaTeX Project Public License, either version 1.3c
% of this license or (at your option) any later version.
% The latest version of this license is in
%    https://www.latex-project.org/lppl.txt
% and version 1.3c or later is part of all distributions of LaTeX
% version 2008 or later.
%
% This file has the LPPL maintenance status "maintained".
%
% The list of all files belonging to the LaTeX base distribution is
% given in the file `manifest.txt'. See also `legal.txt' for additional
% information.
%
% The list of derived (unpacked) files belonging to the distribution
% and covered by LPPL is defined by the unpacking scripts (with
% extension .ins) which are part of the distribution.
%
% \fi
%\iffalse   % this is a METACOMMENT !
%
%% File `oldlfont.dtx'.
%% Copyright (C) 1989-1995 Frank Mittelbach and Rainer Sch\"opf,
%% all rights reserved.
%
%<*dtx>
          \ProvidesFile{oldlfont.dtx}
%</dtx>
%<package>\NeedsTeXFormat{LaTeX2e}
%<package>\ProvidesPackage{oldlfont}
%<driver> \ProvidesFile{oldlfont.drv}
%<*!latex209>
% \fi
%         \ProvidesFile{oldlfont.dtx}
          [2014/09/29 v2.2k Standard LaTeX package]
%
% \iffalse
%</!latex209>
%<*driver>
\documentclass{ltxdoc}
\begin{document}
\DocInput{oldlfont.dtx}
\end{document}
%</driver>
% \fi
%
%
%
%
%
% \providecommand\dst{\expandafter{\normalfont\scshape docstrip}}
%
% \GetFileInfo{oldlfont.dtx}
%
% \title{The file \texttt{oldlfont.dtx} for use with
%        \LaTeXe.\thanks{This file has version
%        number \fileversion, dated \filedate.}\\[2pt]
%        It contains the code for \texttt{oldlfont.sty}}
%
% \author{Frank Mittelbach}
%
% \MaintainedByLaTeXTeam{latex}
% \maketitle
%
% \section{Introduction}
%
%    This file contains the code for the \texttt{oldlfont} package
%    which emulates the following \LaTeX~2.09 font commands:
%    \begin{itemize}
%    \item The two-letter font-changing commands |\rm|, etc.~are
%       defined to cancel each other out as they did in \LaTeX~2.09.
%    \item The two-letter font-changing commands are allowed in math
%       mode.
%    \item The |latexsym| package is loaded.
%    \end{itemize}
%    For full compatibility mode, the file |latex209.def| is loaded by
%    |\documentstyle|.
%
% The following modules are used in the implementation to direct
% \dst{} in generating the external files:
% \begin{center}
% \begin{tabular}{ll}
%   driver   & produce a documentation driver file \\
%   package  & produce |oldlfont.sty| \\
%   latex209 & produce part of |latex209.def|
% \end{tabular}
% \end{center}
%
% \MaybeStop{}
%
% \changes{v2.2k}{1995/11/29}{Remove duplicate driver code.}
%
% \section{The Code}
%
%
% \begin{macro}{\math@bgroup}
% \begin{macro}{\math@egroup}
%   To make \meta{math alphabet identifier} work like simple font
%   switches we change the meaning of |\math@bgroup| and
%   |\math@egroup| to |\@empty|. This emulates the behavior of
% \texttt{oldlfont.sty} in NFSS1 setups.  These definitions are not
%    part of \texttt{latex209} automatically, since to emulate 2.09
%    they are not necessary (all standard fonts are declared as math
%    symbol fonts).
% \changes{v2.2f}{1994/05/05}{Added saved versions of the
% math-groupers, CAR}
% \changes{v2.2g}{1994/05/09}{Moved outside latex209 part}
%    \begin{macrocode}
\let\math@bgroup\@empty
\let\math@egroup\@empty
\let \@@math@bgroup \math@bgroup
\let \@@math@egroup \math@egroup
%    \end{macrocode}
% \end{macro}
% \end{macro}
%
%
%    The rest of this document describes code that is used in
%    |oldlfont.sty| and |latex209.def|.
%    \begin{macrocode}
%<*package|latex209>
%    \end{macrocode}
%
%    When emulating the old settings we don't want a lot of NFSS
%    information being displayed. Thus we required that the
%    \texttt{tracefnt} package is loaded with the option
%    \texttt{errorshow}.
%    \begin{macrocode}
\RequirePackage[errorshow]{tracefnt}
%    \end{macrocode}
%
%    We define math \emph{alphabet} identifiers for the typefaces
%    described in the \LaTeX{} manual.  This is straightforward. Some
%    are already defined by the kernel code.
%    And here are the other ones defined by the old \LaTeX{}. They all
%    get declared as math symbol font alphabets. Thus we first
%    allocate the additional symbol fonts.
%    \begin{macrocode}
\DeclareSymbolFont{bold}{OT1}{cmr}{bx}{n}
\DeclareSymbolFont{sans}{OT1}{cmss}{m}{n}
\DeclareSymbolFont{typewriter}{OT1}{cmtt}{m}{n}
\DeclareSymbolFont{italic}{OT1}{cmr}{m}{it}
\DeclareSymbolFont{smallcaps}{OT1}{cmr}{m}{sc}
\DeclareSymbolFont{slanted}{OT1}{cmr}{m}{sl}
%    \end{macrocode}
%    And here are the corresponding math identifiers.
%    \begin{macrocode}
\DeclareSymbolFontAlphabet\mathbf{bold}
\DeclareSymbolFontAlphabet\mathsf{sans}
\DeclareSymbolFontAlphabet\mathtt{typewriter}
\DeclareSymbolFontAlphabet\mathsc{smallcaps}
\DeclareSymbolFontAlphabet\mathsl{slanted}
%    \end{macrocode}
%    We undefine the old |\mit| and |\cal| macros (whatever meaning
%    they have) and reallocate them as symbol font alphabets.
%    \begin{macrocode}
\let\mit\undefined
\let\cal\undefined
\let\mathit\undefined
\DeclareSymbolFontAlphabet\mathit{italic}
\DeclareSymbolFontAlphabet{\mit}{letters}
\DeclareSymbolFontAlphabet{\cal}{symbols}
%    \end{macrocode}
%
%    We define the font commands for selecting the typeface. They are
%    probably defined by the document class/style but we want to force
%    the old meaning.
%
%    And here are the definition as they were in \LaTeX~2.09 but
%    translated into NFSS2 language.
% \changes{v2.2h}{1994/05/11}{DPC use \cs{DeclareProtectedCommand}}
% \changes{v2.2i}{1994/05/13}{DPC renamed to \cs{DeclareRobustCommand}}
%    \begin{macrocode}
\DeclareRobustCommand\rm{\normalfont\rmfamily\mathgroup\symoperators}
\DeclareRobustCommand\sf{\normalfont\sffamily\mathgroup\symsans}
\DeclareRobustCommand\sl{\normalfont\slshape\mathgroup\symslanted}
\DeclareRobustCommand\bf{\normalfont\bfseries\mathgroup\symbold}
\DeclareRobustCommand\sc{\normalfont\scshape\mathgroup\symsmallcaps}
\DeclareRobustCommand\it{\normalfont\itshape\mathgroup\symitalic}
\DeclareRobustCommand\tt{\normalfont\ttfamily\mathgroup\symtypewriter}
%    \end{macrocode}
%     We also have to define the \emph{emphasize} font change command
%    (i.e.\ |\em|). This command will look whether the current font is
%    sloped (i.e.\ has a positive |\fontdimen1|) and will then select
%    either |\rm| or |\it|.
%    \begin{macrocode}
\DeclareRobustCommand\em{%
  \@nomath\em
  \ifdim \fontdimen\@ne\font>\z@\rm\else\it\fi}
%    \end{macrocode}
%
% \begin{macro}{\@setfontsize}
%    Font size changes are handled using the |\@setfontsize| command
%    (in new class files) or by the |@setsize| command in old document
%    style files. The latter is now defined to call |\@setfontsize| in
%    the NFSS2 kernel code.
%    Thus to make all size changing commands automatically return to
%    the normal font of the document we only have to slightly modify
%    the definition of |\@setfontsize| by adding a |\normalfont|
%    command to it.
% \changes{v2.2j}{1994/11/06}{Use \cs{@typeset@protect}}
%    \begin{macrocode}
\def\@setfontsize#1#2#3{\@nomath#1%
    \ifx\protect\@typeset@protect
      \let\@currsize#1%
    \fi
    \fontsize{#2}{#3}\normalfont}
%    \end{macrocode}
% \end{macro}
%
%
%  \begin{macro}{\non@alpherr}
%    Since we emulate the old syntax we also have to silently ignore
%    uses of a math alphabet outside math mode. Since we now use the
%    |\sym...| switches the following setting is not longer necessary.
%    \begin{macrocode}
%\let\non@alpherr\@gobble
%    \end{macrocode}
%  \end{macro}
%
%  \begin{macro}{\not@math@alphabet}
%    But we need to disable the error message that is generated from
%    |\bfseries| etc.
%    \begin{macrocode}
\let\not@math@alphabet\@gobbletwo
%    \end{macrocode}
%  \end{macro}
%
%
%    We left out the special \LaTeX{} fonts which are not automatically
%    included in the base version of the font selection since these
%    fonts contain only a few characters which are also included in
%    the AMS fonts so anybody who is using these fonts doesn't need
%    them.  But for compatibility reasons we will define these symbols.
%
%    \begin{macrocode}
\RequirePackage{latexsym}
%</package|latex209>
%    \end{macrocode}
%
% \DeleteShortVerb{\|}
% \Finale
\endinput
