% \iffalse meta-comment
%
% Copyright (C) 1993-2023
% The LaTeX Project and any individual authors listed elsewhere
% in this file.
%
% This file is part of the LaTeX base system.
% -------------------------------------------
%
% It may be distributed and/or modified under the
% conditions of the LaTeX Project Public License, either version 1.3c
% of this license or (at your option) any later version.
% The latest version of this license is in
%    http://www.latex-project.org/lppl.txt
% and version 1.3c or later is part of all distributions of LaTeX
% version 2008 or later.
%
% This file has the LPPL maintenance status "maintained".
%
% The list of all files belonging to the LaTeX base distribution is
% given in the file `manifest.txt'. See also `legal.txt' for additional
% information.
%
% The list of derived (unpacked) files belonging to the distribution
% and covered by LPPL is defined by the unpacking scripts (with
% extension .ins) which are part of the distribution.
%
% \fi
% Filename: ltnews04.tex

% This is issue 4 of LaTeX News.

\documentclass
%   [lw35fonts]
   {ltnews}


\publicationmonth{December}
\publicationyear{1995}
\publicationissue{4}

\providecommand\pkg[1]{\texttt{#1}}
\providecommand\cls[1]{\texttt{#1}}
\providecommand\option[1]{\texttt{#1}}
\providecommand\env[1]{\texttt{#1}}
\providecommand\file[1]{\texttt{#1}}

\begin{document}

\maketitle

\section{Welcome to \LaTeXNews~4}

An issue of \emph{\LaTeXNews} will accompany every future release of
\LaTeX.  It will tell you about important events, such as major bug
fixes, newly available packages, or any other \LaTeX{} news.
This issue accompanies the fourth release of \LaTeXe.


\section{\LaTeX\ getting smaller}

The last release in June started a trend of \LaTeX\ becoming
smaller, we are pleased to announce that this has continued with this
release. In particular the experimental `autoload' version described in
\file{autoload.txt} is much smaller as more parts of \LaTeX\ are
autoloaded.

\section{New `concurrent' docstrip}

The time taken to `unpack' this release from the documented sources
should be much reduced (roughly half the time, depending on
installation conditions). This is due to an improved version of the
docstrip program that has been contributed by Marcin Woli\'nski.
This can write up to 16 files at once. The
previous version could only write one file at a time which meant that
it was very slow when producing many small files from the same source
file as the source needed to be re-read for each file written.

\section{New T1 encoded fonts}

This year J\"org Knappen has completed a new release of the `Cork'
(T1) encoded Computer Modern fonts: the dc fonts release 1.2.

This release of the dc fonts fixes many bugs (including the missing
\verb|?`| (?`) and \verb|!`| (!`) ligatures) and improves the fonts in
many other ways. It is strongly recommended that you upgrade as soon as
possible if currently you are using the old dc fonts, release 1.1 or
earlier. The new fonts are available from the CTAN archives, in
\file{tex-archive/fonts/dc}.

The names of the font files are \emph{different}. This does not affect
\LaTeX\ documents but \emph{does} affect the installation procedure as
it assumes that you have the \emph{new} fonts, and will write suitable
`fd' files for those fonts. If you have not yet upgraded your dc fonts
then, after unpacking the distribution, you \emph{must}
\verb|latex olddc.ins| to produce `fd' files for the old dc fonts.
This must be done \emph{before} the format is made. Running the test
document at \file{ltxcheck.tex} the end of the installation will
inform you if the wrong set of `fd' files has been installed.

Note that this change does not affect the standard `OT1' Computer
Modern fonts that \LaTeX\ uses by default.

\section{More robust commands}

The commands \verb|\cite| and \verb|\sqrt| are now robust.

Although most commands with optional arguments are fragile, as
documented, such commands defined using the second optional argument
of \verb|\newcommand| and its derivatives are now \emph{robust}.

\section{New Interface to building `extension' classes}

The mechanism provided by \verb|\DeclareOption|, \verb|\ProcessOptions|
and \verb|\LoadClass| has proved to be a powerful and expressive means
of defining one class in terms of another `base' class. However there
have been some requests to simplify the declaration of the common case
where you want the `base' class to be called with \emph{all} the
options that were specified to the extension class.  This is now
provided by the new command \verb|\LoadClassWithOptions|. A similar
command \verb|\RequirePackageWithOptions| is provided for package use.
More details of this feature are provided in \file{clsguide.tex} and
\file{ltclass.dtx}.

\section{More Input Encodings}

The experimental \pkg{inputenc} package allows a more natural style
of input of accented and other characters.

Three new input encodings are now supported.
\begin{itemize}
\item \option{ansinew} the Windows ansi encoding,
                       as used in Microsoft Windows 3.x.
\item \option{cp437de} a variant of \option{cp437}, which uses \ss\
  rather than $\beta$  in the appropriate slot.
\item \option{next} the encoding used on Next computers.
\end{itemize}

\section{Further information}

For more information on \TeX{} and \LaTeX, get in touch with your local
\TeX{} Users Group, or the international \TeX{} Users Group,
1850 Union Street, \#1637, San Francisco, CA~94123, USA,
Fax:~+1~415~982~8559,
EMail:~tug@tug.org.
The \LaTeX{} home page is \verb|http://www.tex.ac.uk/ctan/latex/|
and contains links to other WWW resources for \LaTeX.

\end{document}
