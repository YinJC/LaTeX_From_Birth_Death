% !TEX program = xelatex
% % % % % % % % % % % % % % % % %
% `ctrl+F` goto 从这里开始
% % % % % % % % % % % % % % % % %

\documentclass{ecustbachelorthesis}
\renewcommand{\thesistype}{(文献翻译)}
\renewcommand{\thesistitle}{华东理工大学本科生毕业论文模板}

\updatecmd
\usepackage{datetime}
\hypersetup{
  pdfinfo={
    Author={dagnaf},
    Title={\thesistitle{}\thesistype},
    CreationDate={D:20150309101424},
    ModDate={D:\pdfdate},
    Keywords={华东理工大学;本科生毕业论文;模板},
    Subject={华东理工大学本科生毕业论文模板}
  }
}

\begin{document}
\label{title:t1}
\pdfbookmark[-1]{文献翻译1}{title:t1}
\mktitle{文献翻译1的标题}{文献翻译1的作者}
\mkabstract{文献翻译1的摘要}{
文献翻译1的关键词
}

\chapter{文献的第一章节}
\begin{chatext}

要删除外文文献中文标题等,可以看到\verb$.tex$代码中有\verb$\makethesistitle[...]...$和\verb$\begin{abstract}[...]...$这两部分。只要删除或者注释即可。注意不要删除论文本身的标题和个人信息。

注意下面的参考文献,如果没有参考文献,则删除\verb$.tex$文件最后的\verb$\nocite{*}$和\verb$\bibliography{...}$。

这里没有引用参考文献,但用于示例,还是显示了\verb$参考文献$的标题。两篇以上的文献翻译在参考文献上有问题,待修改。
% 最后的参考文献,如没有则删除\nocite 和 \bibliography
\end{chatext}
\nocite{*}
\bibliography{t.bib}
\clearpage
\label{title:t2}
\pdfbookmark[-1]{文献翻译2}{title:t2}
\mktitle{文献翻译2的标题}{文献翻译2的作者}
\mkabstract{文献翻译2的摘要}{
文献翻译2的关键词
}
\end{document}
