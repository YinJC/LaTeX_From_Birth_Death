% !TeX TS-program = xelatex

\documentclass[aspectratio=169]{beamer}
\usepackage{etoolbox}
\usepackage{booktabs}

\usepackage{ctex}
\usepackage{fontspec}
\usepackage{tcolorbox}
\usepackage{metalogo}
\usepackage{tasks}
\usepackage{etoolbox}
\usepackage{expl3}
\usepackage{unicode-math}
\usepackage{minted}
\usepackage{xparse}
\usepackage{tikz}

\tcbuselibrary{minted,listings,xparse}

\usefonttheme{professionalfonts}
\usefonttheme{serif}

\setmonofont{IBMPlexMono-Regular}[
    Path=../fonts/,
    Extension=.ttf,
    BoldFont=IBMPlexMono-Bold,
    ItalicFont=IBMPlexMono-Italic,
    BoldItalicFont=IBMPlexMono-BoldItalic
]

\setmainfont{Roboto-Regular}[
    Path=../fonts/,
    Extension=.ttf,
    BoldFont=Roboto-Bold,
    ItalicFont=Roboto-Italic,
    BoldItalicFont=Roboto-BoldItalic
]

\setCJKmonofont{NotoSerifSC-Regular}[
    Path=../fonts/,
    Extension=.otf,
    BoldFont=NotoSerifSC-Bold,
    ItalicFont=NotoSerifSC-Regular
]

\setCJKmainfont{NotoSansSC-Regular}[
    Path=../fonts/,
    Extension=.otf,
    BoldFont=NotoSansSC-Bold
]

\setmathfont{TeX Gyre Termes Math}

\usetheme{AnnArbor}
\usecolortheme{beaver}
\setbeamertemplate{navigation symbols}{}

\author[项子越]{项子越\\ {\scriptsize\ttfamily ziyue.alan.xiang@gmail.com}}
\institute[]{\url{https://github.com/xziyue/latex3-chinese-video}}


% use custom lexer in minted
\newcommand{\pyltlexer}{../tex_lexer.py:Tex3Lexer -x}

\newcommand{\codefontsize}{\fontsize{7}{9}}

\newtcblisting{texcode}{
    listing only,
    listing engine=minted,
    minted options={
        fontsize=\codefontsize,
        linenos,
        autogobble,
        breaklines,
        numbersep=2mm,
        obeytabs,
        tabsize=2
    },
    minted language=\pyltlexer,
    top=0pt,
    bottom=0pt,
    left=4mm,
    colback=white,
    boxrule=1pt
}

\newtcblisting{texcode*}{
    listing engine=minted,
    minted options={
        fontsize=\codefontsize,
        linenos,
        autogobble,
        breaklines,
        numbersep=2mm,
        obeytabs,
        tabsize=2
    },
    minted language=\pyltlexer,
    top=0pt,
    bottom=0pt,
    left=4mm,
    colback=white,
    boxrule=1pt
}

%\newtcblisting{texcode**}{
%    listing engine=minted,
%    minted options={
%        fontsize=\codefontsize,
%        linenos,
%        autogobble,
%        breaklines,
%        numbersep=2mm,
%        obeytabs,
%        tabsize=2
%    },
%    minted language=\pyltlexer,
%    top=0pt,
%    bottom=0pt,
%    left=4mm,
%    colback=white,
%    boxrule=1pt,
%    listing side text
%}



\DeclareTCBListing{texcode**}{!o}{
    listing engine=minted,
    minted options={
        fontsize=\codefontsize,
        linenos,
        autogobble,
        breaklines,
        numbersep=2mm,
        obeytabs,
        tabsize=2
    },
    minted language=\pyltlexer,
    top=0pt,
    bottom=0pt,
    left=4mm,
    colback=white,
    boxrule=1pt,
    listing side text,
    IfValueTF={#1}{
        righthand width=\dimexpr #1\linewidth \relax
    }{}
}


\newtcblisting{progcode}[1]{
    listing only,
    listing engine=minted,
    minted options={
        fontsize=\codefontsize,
        linenos,
        autogobble,
        breaklines,
        numbersep=2mm,
        obeytabs,
        tabsize=2
    },
    minted language=#1,
    top=0pt,
    bottom=0pt,
    left=4mm,
    colback=white,
    boxrule=1pt
}

\renewcommand{\theFancyVerbLine}{\ttfamily \textcolor[rgb]{0.2,0.2.,0.2}{\fontsize{5}{7} \oldstylenums{\arabic{FancyVerbLine}}}}

\newmintinline[texinl]{\pyltlexer}{fontsize=\small}
\newmintinline[textinl]{text}{fontsize=\small}




\title{\LaTeX3教程四:常用库}
\date{2021年12月17日}

\begin{document}

\maketitle

\section{凭据表}

\begin{frame}[fragile]{凭据表 (token list)}
\begin{itemize}
\item 用于存储文档,命令等内容
\item 提供多种使用及修改的接口
\item 可用于存储参数供其它函数调用
\item 主要缺点:无法存储换行符
\end{itemize}
\end{frame}

\begin{frame}[fragile]{凭据表的基本操作}

对凭据表变量赋值
\begin{texcode**}
\ExplSyntaxOn
\tl_gclear:N \g_tmpa_tl %清除旧值
\tl_gset:Nn \g_tmpa_tl {新值}
\cs_meaning:N \g_tmpa_tl
\ExplSyntaxOff
\end{texcode**}

在左侧追加
\begin{texcode**}
\ExplSyntaxOn
\tl_gput_left:Nn \g_tmpa_tl {一个}
\cs_meaning:N \g_tmpa_tl
\ExplSyntaxOff
\end{texcode**}

在右侧追加
\begin{texcode**}
\ExplSyntaxOn
\tl_gput_right:Nn \g_tmpa_tl {被定义}
\cs_meaning:N \g_tmpa_tl
\ExplSyntaxOff
\end{texcode**}

\end{frame}

\begin{frame}[fragile]{凭据表的基本操作}

遍历
\begin{texcode**}
\ExplSyntaxOn
\tl_set:Nn \l_tmpa_tl {ab{cd}\P e}
\tl_map_inline:Nn \l_tmpa_tl {
    \par 元素:#1
}
\ExplSyntaxOff
\end{texcode**}

\end{frame}

\begin{frame}[fragile]{凭据表的基本操作}

按元素序号访问
\begin{texcode**}
\ExplSyntaxOn
\tl_set:Nn \l_tmpa_tl {ab{cd}\P e}
\par\tl_count:N \l_tmpa_tl
\par\tl_item:Nn \l_tmpa_tl {1}
\par\tl_item:Nn \l_tmpa_tl {2}
\par\tl_item:Nn \l_tmpa_tl {3}
\par\tl_item:Nn \l_tmpa_tl {4}
\par\tl_item:Nn \l_tmpa_tl {5}
\ExplSyntaxOff
\end{texcode**}

\end{frame}



\begin{frame}[fragile]{凭据表的基本操作}

取头/取尾
\begin{texcode**}
\ExplSyntaxOn
\tl_set:Nn \l_tmpa_tl {abcde}
\par\tl_head:N \l_tmpa_tl
\par\tl_tail:N \l_tmpa_tl
\ExplSyntaxOff
\end{texcode**}

遍历
\begin{texcode**}
\ExplSyntaxOn
\tl_set:Nn \l_tmpa_tl {ab{cd}\P e}
\tl_map_inline:Nn \l_tmpa_tl {
    \par 元素:#1
}
\ExplSyntaxOff
\end{texcode**}

\end{frame}

\begin{frame}[fragile]{凭据表与字符串}

字符串库 (string)与凭据表十分相似。两者的区别主要在于字符串中所存储的所有内容都会以原样输出。
\begin{texcode**}
\ExplSyntaxOn
\tl_set:Nn \l_tmpa_tl {$\frac{\alpha}{\beta}$}
\tl_use:N \l_tmpa_tl
\ExplSyntaxOff
\end{texcode**}

\begin{texcode**}
\ExplSyntaxOn
\str_set:Nn \l_tmpa_str {$\frac{\alpha}{\beta}$}
\str_use:N \l_tmpa_str
\ExplSyntaxOff
\end{texcode**}

\end{frame}

\begin{frame}[fragile]{中英文数字转换}

\begin{texcode**}
\ExplSyntaxOn
\tl_new:N \g_doc_chn_num_tl
\tl_gset:Nn \g_doc_chn_num_tl {一二三四五六七八九}
\cs_gset:Npn \doc_arabic_to_chn_a:n #1 {
    \tl_item:Nn \g_doc_chn_num_tl {#1}
}
\par\doc_arabic_to_chn_a:n {1}
\par\doc_arabic_to_chn_a:n {3}
\par\doc_arabic_to_chn_a:n {5}
\ExplSyntaxOff
\end{texcode**}

\end{frame}

\begin{frame}[fragile]{中英文数字转换}

\begin{texcode**}[0.1]
\ExplSyntaxOn
\tl_new:N \g_doc_chn_dec_tl
\tl_gset:Nn \g_doc_chn_dec_tl {十百千万}
\cs_gset:Npn \doc_arabic_to_chn:n #1 {
  \tl_set_eq:NN \l_tmpa_tl \g_doc_chn_dec_tl
  \tl_clear:N \l_tmpb_tl
  \doc_arabic_to_chn_b:nN {#1} \l_tmpa_tl     \tl_use:N \l_tmpb_tl
}
\cs_gset:Npn \doc_arabic_to_chn_b:nN #1#2 {
  \int_set:Nn \l_tmpa_int {\int_div_truncate:nn {#1}{10}}
  \int_set:Nn \l_tmpb_int {\int_mod:nn {#1}{10}}
  \int_compare:nNnT {\l_tmpb_int} > {0} {
      \tl_put_left:Nx \l_tmpb_tl {\doc_arabic_to_chn_a:n {\l_tmpb_int}}
  }
  \int_compare:nNnT {\l_tmpa_int} > {0} {
      \tl_put_left:Nx \l_tmpb_tl {\tl_head:N #2}
      \tl_set:Nx #2 {\tl_tail:N #2}
      \exp_args:Nx \doc_arabic_to_chn_b:nN {\int_use:N \l_tmpa_int} #2
  }
}
\ExplSyntaxOff
\end{texcode**}

\end{frame}

\begin{frame}[fragile]{中英文数字转换}

\begin{texcode**}
\ExplSyntaxOn
\par\doc_arabic_to_chn:n {5}
\par\doc_arabic_to_chn:n {15}
\par\doc_arabic_to_chn:n {615}
\par\doc_arabic_to_chn:n {4615}
\par\doc_arabic_to_chn:n {84615}
\par\doc_arabic_to_chn:n {20}
\par\doc_arabic_to_chn:n {320}
\ExplSyntaxOff
\end{texcode**}

\end{frame}

\begin{frame}[fragile]{接收文本参数的格式化}

假设我们想设计一个命令,它能根据一个参数来选择如何格式化文字。比如\texinl|\fmt{cu}{abc}|会将abc加粗;\texinl|\fmt{xie}{abc}|会使用斜体;\texinl|\fmt{xian}{abc}|会使用下划线等。

\begin{texcode**}
\ExplSyntaxOn
\newcommand{\fmt}[2]{
    \str_if_eq:nnT {#1} {cu} {\textbf{#2}}
    \str_if_eq:nnT {#1} {xie} {\textit{#2}}
    \str_if_eq:nnT {#1} {xian} {\underline{#2}}
}
\par\fmt{cu}{abc}
\par\fmt{xie}{abc}
\par\fmt{xian}{abc}
\ExplSyntaxOff
\end{texcode**}

\texinl|\tl_if_eq:|不仅比较字符,还会比较字符的类别码 (category code)。因此可能出现字符看上去相等但是程序判断为不等的情况。

\end{frame}


\begin{frame}[fragile]{接收文本参数的格式化}


\begin{texcode**}[0.2]
\ExplSyntaxOn
\newcommand{\fmt}[2]{
    \str_case:nnTF {#1} {
        {cu} {\textbf{#2}}
        {xie} {\textit{#2}}
        {xian} {\underline{#2}}
    }{
        \textcolor{green}{\ (good)}
    }{
        \textcolor{red}{#2\ (bad)}
    }
}
\par\fmt{cu}{abc}
\par\fmt{xie}{abc}
\par\fmt{xian}{abc}
\par\fmt{123}{abc}
\ExplSyntaxOff
\end{texcode**}

\end{frame}


\section{序列表}


\begin{frame}[fragile]{序列表 (sequence)}

\begin{itemize}
\item 用于存储并管理一系列的元素
\item 可以按序号访问,也可以只访问首元素或尾元素
\item 可以把多个元素连接起来
\end{itemize}

\end{frame}



\begin{frame}[fragile]{序列表的基本操作}

在右侧追加
\begin{texcode**}
\ExplSyntaxOn
\seq_gclear:N \g_tmpa_seq
\seq_gput_right:Nn \g_tmpa_seq {一}
\seq_gput_right:Nn \g_tmpa_seq {个}
\seq_gput_right:Nn \g_tmpa_seq {元素}
\par\seq_use:Nn  \g_tmpa_seq {,~}
\ExplSyntaxOff
\end{texcode**}

在左侧追加
\begin{texcode**}
\ExplSyntaxOn
\seq_gput_left:Nn \g_tmpa_seq {不只}
\par\seq_use:Nn  \g_tmpa_seq {,~}
\ExplSyntaxOff
\end{texcode**}

\end{frame}

\begin{frame}[fragile]{序列表的基本操作}

获取最左侧元素
\begin{texcode**}
\ExplSyntaxOn
\seq_get_left:NN \g_tmpa_seq \l_tmpa_tl
\tl_use:N \l_tmpa_tl
\ExplSyntaxOff
\end{texcode**}

获取最右侧元素
\begin{texcode**}
\ExplSyntaxOn
\seq_get_right:NN \g_tmpa_seq \l_tmpa_tl
\tl_use:N \l_tmpa_tl
\ExplSyntaxOff
\end{texcode**}

\texinl|\seq_pop_left:| 以及 \texinl|\seq_pop_right:|的功能类似,但是会从序列中移除元素。

\end{frame}

\begin{frame}[fragile]{序列表的基本操作}

遍历
\begin{texcode**}
\ExplSyntaxOn
\seq_map_inline:Nn \g_tmpa_seq {
    \par 元素:#1
}
\ExplSyntaxOff
\end{texcode**}

\end{frame}

\begin{frame}[fragile]{序列表的基本操作}

按序号访问
\begin{texcode**}
\ExplSyntaxOn
\par\seq_item:Nn \g_tmpa_seq {1}
\par\seq_item:Nn \g_tmpa_seq {3}
\ExplSyntaxOff
\end{texcode**}

序列表的本质是凭据表
\begin{texcode**}
\ExplSyntaxOn
\cs_meaning:N \g_tmpa_seq
\ExplSyntaxOff
\end{texcode**}

\end{frame}


\begin{frame}[fragile]{序列表的基本操作}

连接序列表中的元素
\begin{texcode**}
\ExplSyntaxOn
\par\seq_use:Nn \g_tmpa_seq {,}
\par\seq_use:Nn \g_tmpa_seq {——}
\ExplSyntaxOff
\end{texcode**}

\end{frame}


\begin{frame}[fragile]{制作正弦表}

我们希望制作如下表格:

\begin{center}
\ExplSyntaxOn
\seq_gclear:N \g_tmpa_seq
\int_step_inline:nnn {0} {18} {
    \seq_put_right:Nn \l_tmpa_seq {
        $\sin #1^\circ = \fp_eval:n {sin(#1 * \c_one_degree_fp)}$
    }
    \int_compare:nNnT {#1} > {0} {
        \int_compare:nNnT {\int_mod:nn {#1 + 1} {3}} = {0} {
            \seq_gput_right:Nx \g_tmpa_seq {
                \seq_use:Nn \l_tmpa_seq {&}
            }
            \seq_clear:N \l_tmpa_seq
        }
    }
} 

\tiny
\begin{tabular}{lll}
\toprule
\seq_use:Nn \g_tmpa_seq {\\}
\\ \bottomrule
\end{tabular}
\ExplSyntaxOff
\end{center}

\end{frame}

\begin{frame}[fragile]{制作正弦表}

\begin{texcode*}
\ExplSyntaxOn
\seq_gclear:N \g_tmpa_seq
\int_step_inline:nnn {0} {18} {
    \seq_put_right:Nn \l_tmpa_seq {
        $\sin #1^\circ = \fp_eval:n {sin(#1 * \c_one_degree_fp)}$
    }
    \int_compare:nNnT {#1} > {0} {
        \int_compare:nNnT {\int_mod:nn {#1 + 1} {3}} = {0} {
            \seq_gput_right:Nx \g_tmpa_seq {
                \seq_use:Nn \l_tmpa_seq {&}
            }
            \seq_clear:N \l_tmpa_seq
        }
    }
} 
\ExplSyntaxOff
\end{texcode*}

\end{frame}

\begin{frame}[fragile]{制作正弦表}

\begin{texcode*}
\ExplSyntaxOn
\tiny
\centering
\begin{tabular}{lll}
\toprule
\seq_use:Nn \g_tmpa_seq {\\}
\\ \bottomrule
\end{tabular}
\par
\ExplSyntaxOff
\end{texcode*}

\end{frame}

\begin{frame}[fragile]{随机列表}

设计两个命令:
\begin{itemize}
\item \texinl|\AddToList|:把元素加入列表
\item \texinl|\ShowList|:以随机顺序输出列表中的元素
\end{itemize}

\begin{texcode*}
\ExplSyntaxOn
\seq_gclear:N \g_tmpa_seq
\cs_gset:Npn \AddToList #1 {
    \seq_gput_right:Nn \g_tmpa_seq {#1}
}
\ExplSyntaxOff
\end{texcode*}

\end{frame}

\begin{frame}[fragile]{随机列表}


\begin{texcode*}
\ExplSyntaxOn
\cs_gset:Npn \ShowList {
    \seq_shuffle:N \g_tmpa_seq
    \begin{itemize}
        \seq_map_inline:Nn \g_tmpa_seq {
            \item ##1
        }
    \end{itemize}
}
\ExplSyntaxOff
\end{texcode*}

\end{frame}


\begin{frame}[fragile]{随机列表}

\begin{texcode**}
\AddToList{两}
\AddToList{只}
\AddToList{老虎}
\AddToList{跑}
\AddToList{得}
\AddToList{快}
\ShowList
\end{texcode**}

\end{frame}

\begin{frame}[fragile]{逗号分隔表 (comma-separated list)}

\begin{itemize}
\item 接口与序列基本相同
\item 区别:可以用逗号分隔的内容来赋值
\end{itemize}

\begin{texcode**}
\ExplSyntaxOn
\clist_set:Nn \l_tmpa_clist {a,b,cd,\P,e}
\clist_map_inline:Nn \l_tmpa_clist {
    \par 元素:#1
}
\ExplSyntaxOff
\end{texcode**}

\end{frame}


\begin{frame}[fragile]{逗号分隔表 (comma-separated list)}


按序号访问
\begin{texcode**}
\ExplSyntaxOn
\clist_set:Nn \l_tmpa_clist {a,b,cd,\P,e}
\par\clist_item:Nn \l_tmpa_clist {1}
\par\clist_item:Nn \l_tmpa_clist {2}
\par\clist_item:Nn \l_tmpa_clist {3}
\par\clist_item:Nn \l_tmpa_clist {4}
\par\clist_item:Nn \l_tmpa_clist {5}
\ExplSyntaxOff
\end{texcode**}

\end{frame}


\begin{frame}[fragile]{接收文本参数的格式化}


\begin{texcode**}[0.2]
\ExplSyntaxOn
\newcommand{\fmt}[2]{
    \group_begin:
    \clist_set:Nn \l_tmpa_clist {#1}
    \clist_map_inline:Nn \l_tmpa_clist {
        \str_case:nn {##1} {
            {cu} {\bfseries}
            {xie} {\itshape}
            {hong} {\color{red}}
        }
    } #2
    \group_end:
}
\ExplSyntaxOff
\par\fmt{cu}{abc}
\par\fmt{xie}{abc}
\par\fmt{hong}{abc}
\par\fmt{cu,hong}{abc}
\par\fmt{xie,hong}{abc}
\par\fmt{cu,xie,hong}{abc}
\end{texcode**}

\end{frame}

\section{属性表}

\begin{frame}[fragile]{属性表 (property list)}

\begin{itemize}
\item 与Python中的字典(dict)类似,提供键—值访问
\end{itemize}

\end{frame}


\begin{frame}[fragile]{属性表的基本操作}

设置值
\begin{texcode**}
\ExplSyntaxOn
\prop_gput:Nnn \g_tmpa_prop {key1} {val1}
\prop_gput:Nnn \g_tmpa_prop {key2} {val2}
\ExplSyntaxOff
\end{texcode**}

取回值
\begin{texcode**}
\ExplSyntaxOn
\par\prop_item:Nn \g_tmpa_prop {key1}
\par\prop_item:Nn \g_tmpa_prop {key2}
\ExplSyntaxOff
\end{texcode**}
取回值的函数还有\texinl|\prop_get:|及\texinl|\prop_pop|等。

\end{frame}

\begin{frame}[fragile]{属性表的基本操作}

遍历
\begin{texcode**}
\ExplSyntaxOn
\prop_map_inline:Nn \g_tmpa_prop {
    \par 键:#1 \quad 值:#2
}
\ExplSyntaxOff
\end{texcode**}
遍历输出的顺序是未定义的。

\end{frame}

\begin{frame}[fragile]{属性表的基本操作}

属性表也可通过类似逗号分隔表的方式初始化。
\begin{texcode**}
\ExplSyntaxOn
\prop_set_from_keyval:Nn \l_tmpa_prop {
    1=一,
    2=二,
    3=三
}

\prop_map_inline:Nn \l_tmpa_prop {
    \par 键:#1 \quad 值:#2
}
\ExplSyntaxOff
\end{texcode**}

\end{frame}

\begin{frame}[fragile]{缩写管理}
设计两个命令:
\begin{itemize}
\item \texinl|\AddAcronym|:用于添加缩写到系统中
\item \texinl|\Acro|:用于取回缩写的全称
\end{itemize}

\begin{texcode**}
\ExplSyntaxOn
\prop_new:N \g_doc_acro_prop
\newcommand{\AddAcronym}[2]{
    \prop_gput:Nnn \g_doc_acro_prop {#1} {#2}
}
\newcommand{\Acro}[1]{
    \textbf{\prop_item:Nn \g_doc_acro_prop {#1}}
}
\ExplSyntaxOff

\AddAcronym{gmm}{高斯混合模型}
\AddAcronym{em}{期望最大化算法}

使用\Acro{em}来优化\Acro{gmm}
\end{texcode**}

\end{frame}



\end{document}