\documentclass[aspectratio=169]{beamer}

\usepackage{ctex}
\usepackage{fontspec}
\usepackage{tcolorbox}
\usepackage{metalogo}
\usepackage{tasks}
\usepackage{etoolbox}
\usepackage{expl3}
\usepackage{unicode-math}
\usepackage{minted}
\usepackage{xparse}
\usepackage{tikz}

\tcbuselibrary{minted,listings,xparse}

\usefonttheme{professionalfonts}
\usefonttheme{serif}

\setmonofont{IBMPlexMono-Regular}[
    Path=../fonts/,
    Extension=.ttf,
    BoldFont=IBMPlexMono-Bold,
    ItalicFont=IBMPlexMono-Italic,
    BoldItalicFont=IBMPlexMono-BoldItalic
]

\setmainfont{Roboto-Regular}[
    Path=../fonts/,
    Extension=.ttf,
    BoldFont=Roboto-Bold,
    ItalicFont=Roboto-Italic,
    BoldItalicFont=Roboto-BoldItalic
]

\setCJKmonofont{NotoSerifSC-Regular}[
    Path=../fonts/,
    Extension=.otf,
    BoldFont=NotoSerifSC-Bold,
    ItalicFont=NotoSerifSC-Regular
]

\setCJKmainfont{NotoSansSC-Regular}[
    Path=../fonts/,
    Extension=.otf,
    BoldFont=NotoSansSC-Bold
]

\setmathfont{TeX Gyre Termes Math}

\usetheme{AnnArbor}
\usecolortheme{beaver}
\setbeamertemplate{navigation symbols}{}

\author[项子越]{项子越\\ {\scriptsize\ttfamily ziyue.alan.xiang@gmail.com}}
\institute[]{\url{https://github.com/xziyue/latex3-chinese-video}}


% use custom lexer in minted
\newcommand{\pyltlexer}{../tex_lexer.py:Tex3Lexer -x}

\newcommand{\codefontsize}{\fontsize{7}{9}}

\newtcblisting{texcode}{
    listing only,
    listing engine=minted,
    minted options={
        fontsize=\codefontsize,
        linenos,
        autogobble,
        breaklines,
        numbersep=2mm,
        obeytabs,
        tabsize=2
    },
    minted language=\pyltlexer,
    top=0pt,
    bottom=0pt,
    left=4mm,
    colback=white,
    boxrule=1pt
}

\newtcblisting{texcode*}{
    listing engine=minted,
    minted options={
        fontsize=\codefontsize,
        linenos,
        autogobble,
        breaklines,
        numbersep=2mm,
        obeytabs,
        tabsize=2
    },
    minted language=\pyltlexer,
    top=0pt,
    bottom=0pt,
    left=4mm,
    colback=white,
    boxrule=1pt
}

%\newtcblisting{texcode**}{
%    listing engine=minted,
%    minted options={
%        fontsize=\codefontsize,
%        linenos,
%        autogobble,
%        breaklines,
%        numbersep=2mm,
%        obeytabs,
%        tabsize=2
%    },
%    minted language=\pyltlexer,
%    top=0pt,
%    bottom=0pt,
%    left=4mm,
%    colback=white,
%    boxrule=1pt,
%    listing side text
%}



\DeclareTCBListing{texcode**}{!o}{
    listing engine=minted,
    minted options={
        fontsize=\codefontsize,
        linenos,
        autogobble,
        breaklines,
        numbersep=2mm,
        obeytabs,
        tabsize=2
    },
    minted language=\pyltlexer,
    top=0pt,
    bottom=0pt,
    left=4mm,
    colback=white,
    boxrule=1pt,
    listing side text,
    IfValueTF={#1}{
        righthand width=\dimexpr #1\linewidth \relax
    }{}
}


\newtcblisting{progcode}[1]{
    listing only,
    listing engine=minted,
    minted options={
        fontsize=\codefontsize,
        linenos,
        autogobble,
        breaklines,
        numbersep=2mm,
        obeytabs,
        tabsize=2
    },
    minted language=#1,
    top=0pt,
    bottom=0pt,
    left=4mm,
    colback=white,
    boxrule=1pt
}

\renewcommand{\theFancyVerbLine}{\ttfamily \textcolor[rgb]{0.2,0.2.,0.2}{\fontsize{5}{7} \oldstylenums{\arabic{FancyVerbLine}}}}

\newmintinline[texinl]{\pyltlexer}{fontsize=\small}
\newmintinline[textinl]{text}{fontsize=\small}




\title{\LaTeX3教程一:简介}
\date{2020年8月25日}

\begin{document}

\maketitle


\section{\LaTeX3的目标}


\begin{frame}[fragile]

\begin{itemize}
\item 提供类似现代编程语言的语法与命名
\item 简化\LaTeX 的宏展开控制
\item 标准化\LaTeX 各个功能的接口
\end{itemize}
\vspace*{1em}
\begin{itemize}
\item \LaTeX3并不是为一般用户而设计的:它是用来实现高级用户接口的一套底层方法。
\end{itemize}

\end{frame}


\subsection{现代化的语法与命名}

\begin{frame}[fragile]



\begin{texcode}
\def\FV@GetLine{\@noligs\expandafter\FV@CheckScan\FancyVerbGetLine}
%% DG/SR modification end
\begingroup
\catcode`\^^M=\active%
\gdef\FancyVerbGetLine#1^^M{%
  \@nil%
  \FV@CheckEnd{#1}%
  \ifx\@tempa\FV@EnvironName%            % True if end is found
    \ifx\@tempb\FV@@@CheckEnd\else\FV@BadEndError\fi%
    \let\next\FV@EndScanning%
  \else%
    \def\FV@Line{#1}%
    \def\next{\FV@PreProcessLine\FV@GetLine}%
  \fi%
  \next}%
\endgroup
\end{texcode}
(节选自\textinl|fancyvrb|)

\end{frame}

\subsection{控制宏展开}

\begin{frame}[fragile]
\begin{columns}

\begin{column}{0.5\linewidth}

\begin{texcode*}
\newcommand{\testcmda}{abc}
\newcommand{\testcmdb}{def\testcmda}
\par\testcmdb
\renewcommand{\testcmda}{def}
\par\testcmdb
\end{texcode*}

\end{column}

\begin{column}{0.5\linewidth}

\begin{progcode}{C++}
string a("abc");

string func_1(string *a){
    // 如果原a的值发生改变,那么返回值也会改变
    return "def" + *a;
}

string func_2(string a){
    // 即使原a的值发生改变,返回值也不会改变
    return "def" + a;
}
\end{progcode}

\end{column}

\end{columns}

\end{frame}


\begin{frame}[fragile]

\begin{texcode*}
\par\uppercase{abcdefghi}
\newcommand{\testcmda}{def}
\par\uppercase{abc\testcmda ghi}
\end{texcode*}

\end{frame}


\subsection{标准化接口}

\begin{frame}[fragile]
\begin{itemize}
\item 整数
\item 浮点数
\item 布尔逻辑
\item 凭据表(token list)
\item 字符串
\item 文件IO
\end{itemize}
\end{frame}

\section{凭据表/字符串}

\begin{frame}[fragile]

我们键入\texttt{tex}文件中的所有内容都可视作凭据表。

\vspace*{1em}

\begin{columns}

\begin{column}{0.3\linewidth}
凭据表内的三类对象:
\begin{itemize}
\item 字符
\item 命令
\item 凭据表
\end{itemize}
\end{column}

\begin{column}{0.7\linewidth}


\texinl|{abc \def {abc\def} ghi}|

\vspace*{2em}

\textinl|{abc \def {abc\def} ghi}|


\end{column}

\end{columns}

\end{frame}

\section{\LaTeX3命名法}

\begin{frame}[fragile]

\LaTeX3命名法提倡将函数的来源以及参数类型、变量的类型编码到其名字内。

\begin{itemize}
\item 可以更方便地让用户区别命令与变量
\item 可以避免不同宏包之间命令的冲突
\item \LaTeX 并不拥有真正的类型系统,这样的命名方式可以让用户在编程时自行检查错误
\item 只是一套指导意见,并不是强制性的要求
\end{itemize}

\end{frame}

\subsection{变量命名法}

\begin{frame}[fragile]
\textinl|\<作用域>_<介绍>_<类型>| \quad \textinl|\<作用域>__<介绍>_<类型>|

\vspace*{1em}

{\footnotesize
\begin{columns}
\begin{column}{0.5\linewidth}
作用域
\begin{itemize}
\item \texttt{l}:局部变量
\item \texttt{g}:全局变量
\item \texttt{c}:常量
\end{itemize}

\vspace*{1em}
{\normalsize
\texinl|\l_tmpa_tl|

\texinl|\g_tmpa_int|

\texinl|\c_left_brace_str|

}
\end{column}
\begin{column}{0.5\linewidth}
类型
\begin{itemize}
\item \texttt{tl}:凭据表
\item \texttt{str}:字符串
\item \texttt{int}:整型
\item \texttt{fp}:浮点数
\item \texttt{seq}:队列
\item \texttt{dim}:尺度/长度
\par\hspace{1cm}\vdots
\end{itemize}
\end{column}
\end{columns}
}


\end{frame}


\subsection{函数命名法}

\begin{frame}[fragile]

\textinl|\<模块>_<介绍>:<参数列表>| \quad \textinl|\__<模块>_<介绍>:<参数列表>|

\vspace*{1em}
常用的参数类型:

{\footnotesize
\begin{columns}
\begin{column}{0.5\linewidth}
\begin{itemize}
\item \texttt{N}:接收一个命令,传递命令本身。
\item \texttt{V}:与\texttt{N}类似,但是传递命令的值。
\item \texttt{n}:接收一个凭据表。
\item \texttt{o}:与\texttt{n}类似,但是对凭据表内的内容进行一次展开。
\end{itemize}
\end{column}
\begin{column}{0.5\linewidth}
\begin{itemize}
\item \texttt{x}:与\texttt{n}类似,但是对凭据表内的内容进行递归展开。
\item \texttt{T}/\texttt{F}:与\texttt{n}类似,用于判断语句中,根据判断结果执行\texttt{T}/\texttt{F}代码。
\item \texttt{c}:接收一个凭据表,返回以其为名字的命令。
\item \texttt{p}:参数列表(\textinl|#1#2|\ldots)
\end{itemize}
\end{column}
\end{columns}
}

\vspace*{1em}

\texinl|\tl_item:Nn|

\texinl|\bool_if:nTF|

\texinl|\tl_put_right:Nx|

\end{frame}


\section{使用\LaTeX3}

\begin{frame}[fragile]

\begin{columns}

\begin{column}{0.5\linewidth}
如何使用\LaTeX3?

\begin{enumerate}
\item \texinl|\usepackage{expl3}|
\item 启用\LaTeX3语法:\texinl|\ExplSyntaxOn|
\item 关闭\LaTeX3语法:\texinl|\ExplSyntaxOff|
\end{enumerate}
\end{column}

\begin{column}{0.5\linewidth}
最小示例

\begin{texcode}
\documentclass{article}
\usepackage{expl3}
\begin{document}
\ExplSyntaxOn
%LaTeX3代码
\ExplSyntaxOff
\end{document}
\end{texcode}

\end{column}

\end{columns}

\end{frame}

\begin{frame}[fragile]

关于\texinl|\ExplSyntaxOn|

\begin{itemize}
\item 所有空格及换行都会被忽略
\item 下划线(\_)和冒号(:)等同于英文字母
\end{itemize}

\end{frame}

\section{简单示例:阶乘}

\begin{frame}[fragile]

\begin{texcode*}
\ExplSyntaxOn
\cs_set:Npn \my_factorial:n #1 {
    \int_set:Nn \l_tmpa_int {1}
    \seq_clear:N \l_tmpa_seq
    \int_step_inline:nn {#1} {
        \seq_put_right:Nn \l_tmpa_seq {##1}
        \int_set:Nn \l_tmpa_int {\l_tmpa_int * ##1}
    }
    $\seq_use:Nn \l_tmpa_seq {\times} = \int_use:N \l_tmpa_int$
}
\par\my_factorial:n {3}
\par\my_factorial:n {7}
\ExplSyntaxOff
\end{texcode*}

\end{frame}

\end{document}