\subsection{平面几何宏包}

%% 一些平面几何使用的宏包\
%% 代码后边不用加;也可以加分号
\begin{tikzpicture}

% 定义多个点(a/b/<name>):坐标(a, b), 名称<name>
\tkzDefPoints{-2.5/0/B, 2.5/0/C, -1.5/3/A, 1.5/3/D, 0/1.5/O}
%% 调用Euclide包中的函数进行操作:求中点, 垂线 ……

%% 定义一个点-->中间不能加空格(A, B)会报错        得到这个点(设为变量E)
\tkzDefMidPoint(A,B)   \tkzGetPoint{E}
\tkzDefMidPoint(C,D)   \tkzGetPoint{F}

%% 连接点
\tkzDrawPolygon(A, B, C, D)

%% 可以使用tikz命令来连接euclide定义的点, 它们二者兼容
\draw (B) -- (F)node[right]{$F$};
\draw (C) -- (E)node[left]{$E$};

%% 以O为中心自动填充标签ABCD。不用加$$
\tkzAutoLabelPoints[center=O](A,B,C,D) 

\end{tikzpicture}

\textbf{一个完美的应用}
\begin{example}
\begin{tikzpicture}[scale=.25]
\tkzDefPoints{00/0/A,12/0/B,6/12*sind(60)/C}
\foreach \density in {20,30,...,240}{%
\tkzDrawPolygon[fill=teal!\density](A,B,C)
\pgfnodealias{X}{A}
\tkzDefPointWith[linear,K=.15](A,B) \tkzGetPoint{A}
\tkzDefPointWith[linear,K=.15](B,C) \tkzGetPoint{B}
\tkzDefPointWith[linear,K=.15](C,X) \tkzGetPoint{C}}
\end{tikzpicture}    
\end{example}

\begin{example}
\begin{tikzpicture}[scale=.75,rotate=-30]
\tkzDefPoint(0,0){O}
\tkzDefPoint(4,-5){A}
\tkzDefIntSimilitudeCenter(O,3)(A,1)
\tkzGetPoint{I}
\tkzExtSimilitudeCenter(O,3)(A,1)
\tkzGetPoint{J}
\tkzDefTangent[from with R= I](O,3 cm)
\tkzGetPoints{D}{E}
\tkzDefTangent[from with R= I](A,1 cm)
\tkzGetPoints{D'}{E'}
\tkzDefTangent[from
with R= J](O,3 cm)
\tkzGetPoints{F}{G}
\tkzDefTangent[from with R= J](A,1 cm)
\tkzGetPoints{F'}{G'}
\tkzDrawCircle[R,fill=red!50,opacity=.3](O,3 cm)
\tkzDrawCircle[R,fill=blue!50,opacity=.3](A,1 cm)
\tkzDrawSegments[add = .5 and .5,color=red](D,D' E,E')
\tkzDrawSegments[add= 0 and 0.25,color=blue](J,F J,G)
\tkzDrawPoints(O,A,I,J,D,E,F,G,D',E',F',G')
\tkzLabelPoints[font=\scriptsize](O,A,I,J,D,E,F,G,D',E',F',G')
\end{tikzpicture}
\end{example}

\begin{example}
\begin{tikzpicture}[scale=.8]
    \tkzDefPoint(3,3){c}
    \tkzDefPoint(6,3){a0}
    \tkzRadius=1 cm
    \tkzDrawCircle[R](c,\tkzRadius)
    \foreach \an in {0,10,...,350}{
    \tkzDefPointBy[rotation=center c angle \an](a0)
    \tkzGetPoint{a}
    \tkzDefTangent[from with R = a](c,\tkzRadius)
    \tkzGetPoints{e}{f}
    \tkzDrawLines[color=magenta](a,f a,e)
    \tkzDrawSegments(c,e c,f)
    }%
\end{tikzpicture}
\end{example}

\begin{example}
\begin{tikzpicture}[scale=.5]
\tkzDefPoint(2,0){A}\tkzDefPoint(0,0){O}
\tkzDefShiftPoint[A](31:5){B}
\tkzDefShiftPoint[A](158:5){C}
\tkzDrawPoints(A,B,C)
\tkzDrawSegments[color = red,
line width = 1pt](A,B A,C)
\tkzProtractor[scale = 1](A,B)
\end{tikzpicture}
\end{example}
