\newcommand{\dis}{\displaystyle}


\section{微分方程的常见解法}
已知有方程
\begin{framed}
    \begin{align*}
        (2x + \frac{1}{y}) \dd x  + (\frac{1}{y^2}-\frac{x}{y^2})\dd y = 0
    \end{align*}      
\end{framed}
    
\subsection{分项组合}
\[\frac{1}{y}\dd x -\frac{x}{y^2}\dd y=\dd\frac{x}{y}+x\dd\frac{1}{y}=2\cdot \dd\frac{x}{y}\] 

当时的错误想法:认为 $y$ 与 $x$ 无关,那么在 $\frac{1}{y}\dd x$ 中,$\frac{1}{y}$ 就是一个常数,常数当然可以拿到微分后边去。
所以后边$\dis \frac{x}{y^2}\dd y$当然也同理,认为 $x$ 与 $y$ 没有关系,于是就产生了这个错误。

\begin{formal}{blue!20}
    ~~认为 $x$ 与 $y$ 无关,它们都有一个微分方程的关系了,那么就可以认为 $y=y(x)$,必然二者有关系。也就是说 $x$ 不能随便的放到后边的 $\dd y$ 中了.
    
    今后$f(x)\dd x$可以把$f(x)$做一个积分,然后放到$\dd x$里边去, $f(y)\dd y$类似,但是遇到形如 $f(x)\dd y$ 或 $f(y)\dd x$的东西的时候
    一定要找前后的搭配
\end{formal}



\textsf{常见的凑微分形式}

\begin{corollary}[常见的凑微分形式]
\begin{align*}
    & x\dd y + y\dd x = \dd(xy)            \hspace*{22em}                          & & \frac{y\dd x-x\dd y}{y^2}=\dd(\frac{x}{y}) \\
    & \frac{y\dd x -xy}{x\dd y} = \dd\bigg[\ln(\frac{x}{y})\bigg]                  & & \frac{y\dd x-x\dd y}{x^2+y^2}=\dd\bigg[{\rm arccot}(\frac{y}{x})\bigg]\\
    & \frac{y\dd x-x\dd y}{x^2-y^2}=\frac{1}{2}\dd\bigg[\ln\biggl|\frac{x-y}{y-x}\biggr|\bigg] & & 
\end{align*} 
\end{corollary}

\subsection{曲线积分}
\noindent{\bf 错解}

因为已知有:$\dis u(x,y)=\int_{x_0}^{x}{M(x,y_0) \dd x}+\int_{y_0}^{y}{N(x,y) \dd y}$.
于是我们取点$(x_0,y_0)=(0, 1)$, 带入上边即有

\begin{align*}
u(x,y)& = \int_{0}^{x}{M(x,1) \dd x}+\int_{1}^{y}{N(x,y) \dd x}
        = \int_{0}^{x}{ 2x+1 \dd x}+\int_{1}^{y}{\bigg(\frac{1}{y}-\frac{x}{y^2}\bigg) \dd y}\\
      & = x^2 + x + C_1 + \ln|y|+\frac{x}{y} + C_2
\end{align*}
所以最终可以知道
\[
    u(x,y)=x^2+x+\ln|y|+\frac{x}{y}+C    
\]


\begin{formal}{blue!20}
    错因: 在$\dis \int_{1}^{y}{\bigg(\frac{1}{y}-\frac{x}{y^2}\bigg)}\dd y$中,当$y=1$时,有一个$\frac{1}{1}-\frac{x}{1}=1-x$, 但是我们忽略了它。
\end{formal}

\noindent{\bf 正解}

首先判断该方程是否为恰当微分方程,有$\dis \frac{\partial M}{\partial y}=\frac{-1}{y}=\frac{\partial N}{\partial x}$,所以该方程是一个恰当的微分方程。同时取点$(0, 1)$,于是就可以得到
\begin{align*}
    u(x,y) =\int_{0}^{x}{\frac{2xy+1}{y}\dd x + \int_{1}^{y}{\frac{y-0}{y^2}} \dd y} 
        = x^2+\frac{x}{y}+\ln|y|\bigg|_{1}^{y}
        = x^2+\frac{x}{y}+\ln|y|+C
\end{align*} 

\textsf{曲线积分方法}

\begin{theorem}[曲线积分法]
$
    \begin{aligned}
        \text{标准形式:}M(x,y)\dd x+N(x,y)\dd y=\frac{\partial u}{\partial x}\dd x&+\frac{\partial u}{\partial y}\dd y\\
        &\text{标准解1:}u(x,y)=\int_{x_0}^{x}{M(x,y_0)\dd x} + \int_{y_0}^{y}{N(x,y) \dd y}\\
        &\text{标准解2:}u(x,y)=\int_{y_0}^{y}{N(x_0,y)\dd y} + \int_{x_0}^{x}{M(x,y) \dd x}\\
    \end{aligned}
$ 

\textsf{理解}


~~~~我们可以认为是先从$(x_0, y_0)\to(x, y_0)$积分,因为此时只有$x$在变化,$y$是常数,恒为$y_0$,
所以这个值就是$\frac{\partial }{\partial x}\int_{y_0}^{y}{M(x,y_0) \dd y}$.
然后第二项就是从$(x, y_0)\to (x, y)$,此时只有$y$在变化,$x$为常数,恒为$x$,
所以第二项就是$\int_{y_0}^{y}{N(x, y) \dd x}$.

所以综上我们就可以得到:

\begin{align}
    u(x,y)=\int_{x_0}^{x}{M(x,y_0) \dd x} + \int_{y_0}^{y}{N(x,y) \dd y}
\end{align}
\end{theorem}

\begin{proof}

我们从$\frac{\partial u}{\partial y}=N(x,y)$入手,可以很容易的得到下面的式子:

\begin{align*}
    u(x,y)&=\int_{y_0}^{y}{N(x,y) \dd y}+\psi(x)
        \Longrightarrow \frac{\partial}{\partial x}[\int_{y_0}^{y}{N(x,y) \dd y}]+\psi'(x)=M(x,y) \\
    \Longrightarrow  \psi'(x) 
    & = M(x,y)-\frac{\partial}{\partial x}\bigg(\int_{y_0}^{y}{N(x,y) \dd y}\bigg)+\psi'(x)
      = M(x,y)-\int_{y_0}^{y}{\frac{\partial}{\partial x}N(x,y) \dd y}\\
    & = M(x,y)-\int_{y_0}^{y}{M(x,y) \dd y}
      = M(x, y)-\bigg(M(x,y)-M(x,y_0)\bigg)
\end{align*} 
带入最上边的式子即可得
\[
    u(x,y)=\int_{y_0}^{y}{N(x,y) \dd y}+\int_{x_0}^{x}{M(x,y_0) \dd x}    
\]
\end{proof}

\textsf{错误记忆方式}

\begin{align*}
    u(x,y) = \int_{}^{}{M(x,y_0) \dd x}+\int_{}^{}{N(x,y) \dd y}
           = \int_{}^{}{M(x,y) \dd x}+\int_{}^{}{N(x,y_0) \dd y}
\end{align*} 


\begin{formal}{blue!20}
    这种记忆方法完全是错的,就像上边的错误1,这样会漏掉常数项对应的$x,y$的多项式
\end{formal}


\clearpage
\subsection{判定定理}
\begin{proposition}[判定定理]
\begin{align}   
    u(x, y) = \int_{}^{}{M(x, y) \dd x} + \int_{}^{}{[N(x, y)-\frac{\partial}{\partial y}\int_{}^{}{M(x, y) \dd x}] dy} 
\end{align}         
\end{proposition} 


\subsection{积分因子}
\begin{framed}

\noindent{\sf 作用:把非恰当的微分方程转化为恰当的微分方程}

$xdy-ydx=0$,很容易看出这不是一个恰当微分方程,因为$\frac{\partial M(x, y)}{\partial y}=-1\neq \frac{\partial N(x, y)}{\partial x}=1$。
这个时候我们在两边同时乘以$\frac{1}{y^2}$就可以得到:$\frac{1}{y}\dd x-\frac{x}{y^2}\dd y=0$.
即$\dd(\frac{x}{y})=0\Longrightarrow \frac{x}{y}=C$。

关于怎么凑积分因子:可以记住常见的二元函数的微分形式, 可以参见前面常见的\textsf{常见的凑微分形式}小节。
\end{framed}

\vspace*{2em}
\noindent{\sf 积分因子的推导}

背景:如果存在连续可微的函数$u(x, y)$,使得 
\begin{align*}
    u(x, y)M(x, y)\dd x + u(x, y)N(x, y)\dd y = 0
\end{align*}
恰好为恰当微分方程。则称$u(x, y)$为方程的积分因子。

根据恰当微分方程的判定定理可以知道,函数$u(x, y)$为不恰当微分方程的解的\textsf{充要条件}为:
\begin{align*}
    \frac{\partial (\mu M)}{\partial y}=\frac{\partial (\mu N)}{\partial x} 
    \Longrightarrow  
    N  \cdot \frac{\partial \mu}{\partial x}-M\cdot \frac{\partial \mu}{\partial y} = [\frac{\partial M}{\partial x} - \frac{\partial N}{\partial x}]
\end{align*}

这是一个一阶偏微分方程,直接求解 $\mu(x, y)$ 是很复杂的。所以我们一般考虑积分因子为单变量函数的情形。
即  $\mu= \mu(x)$ 或 $\mu = \mu(y)$. 如果 $\mu$ 只与$x$($y$同理)有关,那么就有
\[
    \frac{\partial \mu}{\partial y} = 0    
\]
原来的偏微分方程就可以简化为
\[
    N\cdot \frac{\dd\mu}{\dd x} = \biggl(\frac{\partial M}{\partial y}-\frac{\partial N}{\partial x}\biggr)\cdot\mu    
\]
\begin{align*}
    \Longrightarrow \frac{\dd\mu}{\mu} = \frac{\frac{\partial M}{\partial y}-\frac{\partial N}{\partial x}}{N}\dd x
\end{align*}
得出一个结论
\[
    \frac{1}{N} \biggl(\frac{\partial M}{\partial y}-\frac{\partial N}{\partial x}\biggr) = \psi(x) 
    \Longleftrightarrow 
    \mu (x) = \mathrm{Exp}\{\int_{}^{}{\psi (x) \dd x}\}   
\]

\bigskip
\begin{corollary}[寻找积分因子] 
只需要去尝试以下的两个式子即可,看能否只和一个变量有关
\begin{align}
    \frac{\frac{\partial M}{\partial y}-\frac{\partial N}{\partial x}}{N}=\psi(x) \hspace*{0.5\linewidth} \frac{\frac{\partial M}{\partial y}-\frac{\partial M}{\partial x}}{-M}=\varphi(x)
\end{align} 
\end{corollary}



\clearpage
\subsection{一阶隐式微分方程}
一阶隐式微分方程的一般形式为
\begin{align}
    F(x, y, y') = 0\nonumber
\end{align}
但是有一般形式的隐式方程解决的难度过大, 且没有通用的解决方法,于是下边主要针对它化简为的4种形式进行讨论。

\begin{framed}
    \begin{align*}
        y = f(x, y') \hspace*{8em} y = f(y, y') \hspace*{8em} F(x, y')= 0 \hspace*{8em} F(y, y')= 0
    \end{align*}
\end{framed}

\begin{formal}{blue!20}
  申明一个点,其实上边只有两种情况,毕竟$x, y$是等价的。  
\end{formal}


\noindent{\sf 第一种形式}

令 $y' = p$, 于是
\[
    \frac{\partial y}{\partial x} = \frac{\partial }{\partial x}f(y, p)
\]
可以得到
\[
p = \frac{\partial f}{\partial x}+\frac{\partial f}{\partial p}\frac{\dd p}{\dd x}
\]
所以 $\Phi(x, p, c)=0$,  从而方程的解为
\begin{align*}
    \left\{
        \begin{aligned}
            & \Phi(x, p, c) = 0 \\
            & y = f(x, p)
        \end{aligned}
    \right.
\end{align*}

\begin{formal}{blue!20}
    注意:$f(x, p)$一定要使用多元函数求导法则进行求导
\end{formal}

\noindent{\sf 第二种形式}

也就是形如$F(x, y')= 0$的方程的求解, 和第一种思想相似, 也是令$y'= p$,所以就有$F(x, p) = 0 $。
从几何意义上说它就是平面$Oxp$上的一条曲线, 所以它不然可以用一个如此下的参数方程表示
\begin{align*}
    \left\{
        \begin{aligned}
            & x = \varphi(t)\\
            & p = \psi(t)
        \end{aligned}
    \right.
\end{align*}

从参数方程的求导法则可以知道,
\begin{align*}
    \frac{\dd y}{\dd x} = {\frac{\dd y}{\dd t}}\bigg/{\frac{\dd x}{\dd t}} = p 
    \Longrightarrow  \frac{\dd y }{\dd t } &= p\cdot \frac{\dd x }{\dd t } = \psi(t) \varphi'(t)
    \Longrightarrow  y = \int_{}^{}{\psi (t ) \varphi' (t)\dd t}
\end{align*}


\clearpage
\noindent{\sf 一个例子}

\begin{framed}
    \begin{align*}
        x^3 + y'^3 - 3 xy'= 0.\hspace*{15em}\text{求解}\quad \frac{\dd y }{\dd x }
    \end{align*}
\end{framed}


\begin{proof}

    首先令 $y' = p = tx$, 带入原始的中得到(备注:这里实际上是设$x =  x(t)\Rightarrow  y = t\cdot x(t) \Rightarrow y = y(t)$)
    \[
        x = \frac{3t }{1+t^3} \Longrightarrow p = \frac{3t^2}{ 1+t^3}
    \]
    这里不能认为$y = \int_{}^{}{p  dt}$,因为我们求的是$y=y(t)$ 但是这里是$y = y(t)$.
    也就更谈不上把$y = y(t)$解出来以后,反解$t = t(x)$带入$p = \frac{3t^2}{1+t^3}$从而得到$y = y(x)$了;
    于是可以得到
    \[
        \dd y = \frac{9(1-2t^3)t^2}{(1+t^3)^3}\dd t
    \]
    这里就是得到$y = y(t)$的表达式。因为前边$x= x(t)$的表达式也求出来了; 
    积分可得
    \begin{align*}
        y = \int_{}^{}{\frac{9(1-2t^3)t^2}{(1+t^3)^3} dt} = \frac{3}{2}\cdot \frac{1+4t^3}{(1+t^3)^2} + C
    \end{align*}

    原方程的解为
    \begin{align*}
        \left\{
        \begin{aligned}
            & x= \frac{3t }{1+t^3}\\ 
            & y = \frac{3}{2}\cdot \frac{1+4t^3}{(1+t^3)^2} + C
        \end{aligned}
        \right.
    \end{align*}
\end{proof}


\clearpage
\noindent{\sf 关于积分因子的一个问题}

试证明\textsf{齐次}微分方程$M(x, y)dx + N(x, y)dy = 0$,当$xM + yN \neq 0$时有积分因子
\[
    \mu = \frac{1}{xM + y N}
\]



\begin{proof}

只需要证明如下的等式成立即可
\begin{align*}
    &\frac{\partial }{\partial y}\frac{M}{xM+yN}=\frac{\partial }{\partial x}\frac{N}{xM+yN}\\
    & {\rm LHS} =\frac{\partial}{\partial y}\left(\frac{M}{x M+y N}\right)=\frac{M_y(x M+y N)-\left(x M_y+N+y N_y\right) \cdot M}{(x M+y N)^2}\\
    & {\rm RHS}=\frac{\partial}{\partial x}\left(\frac{N}{x M+y}\right)=\frac{N_x(x M+y N)-\left(M+x M_x+y N_x\right)\cdot N}{(x M+y N)^2}\\
\end{align*}

即证
\[
    M_y(x M+y N)-\left(x M_y+N+y N_y\right) \cdot M=N_x(x M+y N)-\left(M+x M_x+y N_x\right) \cdot N
\]

也就是证明如下的等式成立
\begin{align*}
    (M_y-N_x)(x M+y N)&=M(x M_y+N+y N_y)-N\cdot(M+x M_x+y N_x)\\
    &=M x \cdot M_y+{M N}+y M N_y-{N M}-x N M_x-y N N_x\\
    &=M x \cdot M_y+y M N_y-x N M_x-y N N_x\\
    & = {\rm RHS}
\end{align*}

我们可以进行一系列的化解操作,具体的化简步骤如下所示
\[
\Longrightarrow {\rm LHS}= \cancel {x M M_y} - \cancel{y N N_x} + y N M_y - x M N_x = \cancel{ x M \cdot M_y} + y M N_y-x N M_x- \cancel{y N N_x} = {\rm RHS}
\]

即证 
\[
    y N M_y - x M N_x = y M N_y - x N M_x
\]

以下就是两个思考方向:
\begin{align*}
    &\Rightarrow 1^0 \quad y(N\cdot M_y - M\cdot N_y) = x(M\cdot N_x - N\cdot M_x)\\
    &\Rightarrow 2^0 \quad M\left(x N_x+y N_y\right)=N\left(x M_x+y M_y\right)
\end{align*}

\end{proof}

\begin{definition}[齐次函数]
    \noindent\textsf{定义}

    如果把函数$f(x_1, x_2, \cdots , x_n)$的每一个自变量乘以一个因子$\lambda$, 如果此时因变量相当于原函数
    乘以这个因子$\lambda$的幂,则称此函数为齐次函数。
    定义$f(x_1, x_2, \cdots , x_n)$为$k$次齐次函数,需满足关系:
    \begin{align*}
    f(\lambda x_1, \lambda x_2, \cdots , \lambda x_n) = \lambda^{k} f(x_1, x_2, \cdots , x_n)
    \end{align*} 

    \noindent \textsf{欧拉定理}

    对于$k$次齐次函数$f(x_1, x_2, \cdots , x_n)$, 有齐次函数的欧拉定理:
    \begin{align}
    x_1 \frac{\partial f}{\partial x_1} + x_2 \frac{\partial f}{\partial x_2} + \cdots + x_n \frac{\partial f}{\partial x_n} = kf(x_1, x_2, \cdots , x_n)
\end{align}
\noindent\textsf{齐次方程}

如果方程
\[\frac{\dd y}{\dd x}= f(x, y)\]
的右端函数$f(x, y)$为它的变量的\textsf{零次}齐次函数,即满足恒等式
\[f(tx, ty) = f(x, y)\]
那么上述方程称为齐次方程.
\end{definition}


\begin{proof}
    因为函数$f(x_1, x_2, \cdots , x_n)$为$k$次齐次函数, 所以对定义式两边全微分有:

    \begin{align*}
        \frac{\partial }{\partial \lambda} {\rm LHS} 
            & =\frac{\dd}{\dd\lambda}f(\lambda x_1, \lambda x_2, \cdots , \lambda x_n)
              = \sum_{i=1}^{n}{\frac{\partial f}{\partial (\lambda x_i)}\frac{\dd (\lambda x_i)}{\dd \lambda}}
              = \sum_{i=1}^{n}{x_i \frac{\partial f}{\partial (\lambda x_i)}} \\
            & = \frac{\partial }{\partial \lambda} {\rm RHS}
              = \frac{\dd}{\dd \lambda} \lambda^k f( x_1,  x_2, \cdots ,  x_n) 
              = k \lambda^{k-1}f( x_1,  x_2, \cdots ,  x_n)\\
            & \Longrightarrow \sum_{i=1}^{n}{x_i \frac{\partial f}{\partial (\lambda x_i)}}  = k \lambda^{k-1}f( x_1,  x_2, \cdots ,  x_n)\\
    \end{align*} 

    令, $ \lambda = 1$ 我们即有
    \begin{align*}
        \sum_{i=1}^{n}{x_i \frac{\partial f}{\partial (x_i)}} & = k f( x_1,  x_2, \cdots ,  x_n)
    \end{align*} 
    

    \noindent{\sf 齐次多项式}
    
    由同次数的单项式相加得到的多项式, 比如$x^5 + 2x^3y^2 + 9x\cdot 4y - 4 y^5$是5次齐次多项式(函数)
\end{proof}


\noindent{上述(思考方向$1^\circ$)完善}
\[
    {\rm LSH} = k M N(x, y) = k M(x, y) N = {\rm RSH}, \qquad k = 1
\]

\begin{proof}\num{\mbox{参考证明}}

设 $M(x, y), N(x, y)$是$m$次齐次函数, 令 $y = ux, u = u(x) \Rightarrow dy = xdu + u dx$, 我们可以得到
\begin{align*}
    & M(x, y) = M (x, ux) = x^m M(x, y)\\
    & N(x, y) = N (x, ux) = x^m N(x, y)   
\end{align*}

方程两边同时乘以乘积因子, 
\[
    \mu  = \frac{1}{xM +yN}(xM + yN \neq 0)
\]

得到
\[
    \frac{M}{xM + yN} dx + \frac{N}{xM + yN}dy = 0
\]

把 $\dd y = x\dd\mu  + \mu \dd x$ 带入上式可以得到
\[
    \frac{M(x,\mu)+ \mu N(x, \mu x)}{xM(x, \mu x)+ yN(x, \mu x)}\dd x + \frac{xN(x, \mu x)}{xM(x, \mu x)+ yN(x, \mu x)}\dd\mu = 0
\]

展开化解为
\[
    \frac{x^m M(1, \mu)+ ux^m N(1, \mu)}{x^{m+1} M(1, \mu)+ ux^{m+1} N(1, \mu)}\dd x + \frac{x^{m +1} N(1, \mu )}{x^{m+1} M(1, \mu)+ ux^{m+1} N(1, \mu )}\dd\mu = 0
\]

进一步化简即得
\[
    \frac{1}{x}\dd x + \frac{N(1, \mu)}{M(1, \mu)+ \mu N(1, \mu)}\dd \mu = 0
\]

进行恰当微分方程得检验可知
\[
    \frac{\partial }{\partial \mu} \left[\frac{1}{x}\right] = 0 = \frac{\partial }{\partial x}\left[\frac{N(1, \mu)}{M(1, \mu)+ \mu N(1, \mu)}\right]    
\]

故上述方程为恰当微分方程
\begin{align*}
   \Longrightarrow \mu  = \frac{1}{xM +yN},(xM + yN \neq 0)\quad\text{是积分因子}
\end{align*} 
\end{proof}

\begin{formal}{blue!20}
    从上边的解答过程中还可以得到了方程: 
    \[
        M(x, y)\dd x + N(x, y)\dd y=0, (xM+ yN \neq 0)
    \]
    的解为
    \[
        x = C\cdot{\rm Exp}\{-\int_{}^{}{\frac{N(1, \mu)}{M(1, \mu)+ \mu N(1, \mu)}\dd \mu}\}
    \]
    提示:
    \[
        \frac{1}{x}\dd x + \frac{N(1, \mu)}{M(1, \mu)+ \mu N(1, \mu)}\dd \mu = \dd\left[\frac{1}{x}\right] + \dd\left[\int_{}^{}{\frac{N(1, \mu)}{M(1, \mu)+ \mu N(1, \mu)} \dd \mu}\right] = 0
    \]   
\end{formal}
