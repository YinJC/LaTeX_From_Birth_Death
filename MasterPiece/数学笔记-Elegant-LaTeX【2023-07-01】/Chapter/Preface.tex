\begin{quote}
    \zihao{-5}\kaishu
    \centerline{数学之诗}
    拉格朗日,傅立叶旁, 我凝视你凹函数般的脸庞。 

    微分了忧伤, 积分了希望, 我要和你追逐黎曼最初的梦想。

    感情已发散,收敛难挡, 没有你的极限,柯西抓狂。

    我的心已成自变量, 函数因你波起波荡。

    低阶的有限阶的, 一致的不一致的, 是我想你的皮亚诺余项。

    狄利克雷,勒贝格、杨 , 一同仰望莱布尼茨的肖像, 拉贝、泰勒,无穷小量, 是长廊里麦克劳林的吟唱。

    打破了确界,你来我身旁, 温柔抹去我,阿贝尔的伤。

    我的心已成自变量, 函数因你波起波荡。

    低阶的有限阶的, 一致的不一致的, 是我想你的皮亚诺余项。

    欧几里德留下了几何原本,传抄在雪白的羊皮纸上,距今已有两千三百多年;     

    阿波罗尼生于帕加,凝视着永恒的圆锥曲线; 

    丢番图却在静静的欣赏不定方程的解,微分、级数、离散、收敛是谁的发现?         

    喜欢你在连续之中逼近我的极限,经过剑桥三一学院,

    我以牛顿之名许愿,思念就像傅利叶级数一样蔓延,

    当空间只剩下拓扑的语言,映射就成了永垂不朽的诗篇,

    我给你的爱写在Banach空间,深埋在康托尔集合里面,

    用超越数去超越永远,那一绝对收敛的数列,一万年都不变.
    \end{quote}

    \vspace*{0.3\linewidth}
    \begin{figure}[!htb]
    \begin{minipage}[t]{0.4\linewidth}
        \mbox{}
    \end{minipage}
    \hfill
    \begin{minipage}[t]{0.5\linewidth}
        \begin{QuoteEnv}[法国教育改革笑话]{}
           \noindent{一个法国人问一个小学生:你知道2+3等于多少吗?}\\
            小学生:不知道。\\
            法国人:那3+2呢?\\
            小学生:不知道。\\
            法国人:那你知道什么?\\
            小学生:我知道3+2等于2+3。\\
            法国人:为什么?\\
            小学生:因为这是一个阿贝尔群。\\
        \end{QuoteEnv}        
    \end{minipage}
    \end{figure}