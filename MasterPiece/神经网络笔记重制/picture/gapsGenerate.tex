% 生成最终的图象时把第一个文档类取消注释即可
\documentclass[10pt,varwidth]{standalone}
% \documentclass[12pt]{article}
% 1.必须添加varwidth选项,不然就会报错
\PassOptionsToPackage{quiet}{fontspec}
\usepackage{ctex}
\usepackage{geometry}
% 必须要保证绘图的纸张足够的大
\geometry{a2paper,left=1in,right=1in,top=1in,bottom=1in}
\usepackage{xifthen}
\usepackage{xfp}
\usepackage{xcolor}
\usepackage{pgfplots}
\usepackage{pgfplotstable}
\pgfplotsset{compat=1.16}
% 2.引用的tikz库
\usetikzlibrary {matrix, chains, trees, decorations}
\usetikzlibrary {arrows.meta, automata,positioning}
\usetikzlibrary {decorations.pathmorphing, calc}
\usetikzlibrary {backgrounds, mindmap,shadows}
\usetikzlibrary {patterns, quotes, 3d, shadows}
\usetikzlibrary {graphs, fadings, scopes}
\usetikzlibrary {arrows, shapes.geometric}
\usepgflibrary {shadings}

\tikzset{
    >={Latex[length=6mm, width=2mm]}
}



\begin{document}
\begin{tikzpicture}
    \node(A11) at (0, 0) {千米:100};
    \node(A21) at (5, 0) {\parbox{6em}{英里\\=千米 $\times$ 0.61}};
    \node(A22) at (5, -5) {误差:1.137};
    \node(A31) at (10, 0) {\parbox{8em}{计算英里数:\\\hspace*{2em} 61}};
    \node(A32) at (10, -2) {\parbox{8em}{正确英里数:\\\hspace*{1em} 62.137}};

    % 2.基本架构
    \draw[draw=black] (A21) circle(4em);
    \draw[->, thick] (A11) -- (A21);
    \draw[->, thick] (A21) -- (A31);
    \draw[->, thick] (A22) -- ($(A32) + (-3em, 0em)$);
\end{tikzpicture}
\end{document}