% % 生成最终的图象时把第一个文档类取消注释即可
% \documentclass[10pt,varwidth]{standalone}
% % \documentclass[12pt]{article}
% % 1.必须添加varwidth选项,不然就会报错
% \PassOptionsToPackage{quiet}{fontspec}
% \usepackage{ctex}
% \usepackage{geometry}
% % 必须要保证绘图的纸张足够的大
% \geometry{a2paper,left=1in,right=1in,top=1in,bottom=1in}
% \usepackage{xifthen}
% \usepackage{xfp}
% \usepackage{pgfplots}
% \usepackage{pgfplotstable}
% \pgfplotsset{compat=1.16}
% % 2.引用的tikz库
% \usetikzlibrary{matrix,chains,trees,scopes,decorations}
% \usetikzlibrary{arrows.meta,automata,positioning,shadows,3d}
% \usetikzlibrary {decorations.pathmorphing}
% \usetikzlibrary {backgrounds,mindmap,shadows}
% \usetikzlibrary {patterns, quotes}
% \usetikzlibrary {graphs,fadings}
% \usetikzlibrary {arrows,shapes.geometric, calc}
% \usepgflibrary {shadings}



% \tikzset{
%     >={Latex[length=6mm, width=2mm]}
% }


% 生成最终的图象时把第一个文档类取消注释即可
\documentclass[10pt,varwidth]{standalone}
% \documentclass[12pt]{article}
% 1.必须添加varwidth选项,不然就会报错
\PassOptionsToPackage{quiet}{fontspec}
\usepackage{ctex}
\usepackage{geometry}
% 必须要保证绘图的纸张足够的大
\geometry{a2paper,left=1in,right=1in,top=1in,bottom=1in}
\usepackage{xifthen}
\usepackage{xfp}
\usepackage{pgfplots}
\usepackage{pgfplotstable}
\pgfplotsset{compat=1.16}
% 2.引用的tikz库
\usetikzlibrary {matrix, chains, trees, decorations}
\usetikzlibrary {arrows.meta, automata,positioning}
\usetikzlibrary {decorations.pathmorphing, calc}
\usetikzlibrary {backgrounds, mindmap,shadows}
\usetikzlibrary {patterns, quotes, 3d, shadows}
\usetikzlibrary {graphs, fadings, scopes}
\usetikzlibrary {arrows, shapes.geometric}
\usepgflibrary {shadings}

\tikzset{
    >={Latex[length=6mm, width=2mm]}
}



\begin{document}
\begin{tikzpicture}[->]
    
    % 1.声明node
    \foreach \x/\name in {1, 2/i, 3/j, 4/k, 5}{
        % 1.批量命名所有的节点,产生从A11~A53共15个标记node
        \foreach \y in {1,2,3}{
            \ifnum\x>1 \ifnum\x<5
                \ifnum\y >1
                    \node at (\x*4, {-(\y-1)*4}) (A\x\y)%
                        {$\name\ =\ \fpeval{\x-1}$};
                \fi 
            \fi\fi
            \node at (\x*4, {-(\y-1)*4}) (A\x\y) {}; % 这里必须添加{},不然会解析错误
        }
    % 2.批量连接第3,4层节点
        \ifnum\x < 5 
            \ifnum\x > 2
                \foreach \y in {2,3}{
                    \foreach \nexty in {-4, -8}{
                        % shorten <= 1.5em 和 shorten >= 1.5em 分别将箭头的起点和终点缩短了 1.5em
                        \draw[thick, shorten <= 1.5em, shorten >= 1.5em]%
                             (4*\x-4, -4*\y+4) -- (4*\x, \nexty);
                    }
                }
            \fi 
        \fi
    }

    % 3.给2,3,4层节点设置样式
    \foreach \x in {2, 3, 4} {
        \foreach \y in {2,3}{
            \draw[thick] (A\x\y) circle(1);
        }
    }

    % 4.补充层名
    \foreach \ceilnum/\ceilname in {1/输入层, 2/隐藏层, 3/输出层}{
        \node[below=3em] at (A\fpeval{\ceilnum+1}1) {\ceilname\ceilnum};
    }
    \node[below=5.5em, blue, thick] at (A51) {
        \parbox{14em}{\centering 节点误差=目标值-实际值\\ $e_k=t_k-o_k$}
    }; 
    \node[below=14em, blue, thick] at (A51) {输出层:$o_k$};
    \draw[thick, shorten >= 1.5em] (A12) -- (A22);
    \draw[thick, shorten >= 1.5em] (A13) -- (A23);
    % 不想用draw,但是使用node 也不方便时,便可以考虑使用\path命令
    \path (A12) -- node[pos=.5, above] {输入层} (A13);
    \draw[thick, shorten <= 1.5em] (A42) -- (A52);
    \draw[thick, shorten <= 1.5em] (A43) -- (A53);
    \foreach \n in {2, 3, 4}{
        \draw[opacity=0.5, fill=orange!50!white, draw=none] (A\n2)+(-1.25, 3) rectangle ($(A\n3) + (1.25, -1.25)$);  
    }


    % 5.图注
    % ($(node) + (x_add, y_add)$): 表示一个点的偏移点(需要使用tikz的calc库)
    % 当然对于起始点的偏移有更简单的方法: (node)+(x_add, y_add)
    \node[blue] (B) at (15, -9.5) {\parbox{12em}{\centering 隐藏层节点输出:$N$}};
    \node[green!70!orange] (C) at (14, -1.5) {\parbox{6em}{\centering 权重:$W_{j,k}$}};
    \draw[->, very thick, blue] (B) -- ($(A33) + (2.5em, 0em)$);
    \draw[->, very thick, blue] (B) -- ($(A32) + (2em, -1.9em)$);
    \draw[->, very thick, green!70!orange] (C) -- ($(A32) + (3em, -3em)$);
    \draw[->, very thick, green!70!orange] (C) -- ($(A32) + (6em, 0em)$);

\end{tikzpicture}
\end{document}