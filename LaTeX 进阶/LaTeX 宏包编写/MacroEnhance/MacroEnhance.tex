\documentclass[fontset=windows, 12pt]{article}
\usepackage[test]{Article}
% \usepackage[test]{CmdAndEnv}


\newcommand{\baa}[1]{I eat #1{}. }
%% <==>
\NewDocumentCommand{\bab}{m}{I eat #1{}. }
% m表示\bab命令接受一个标准的latex参数(必须给出的参数)
% g表示该参数是一个可选参数,并且用{}界定其范围。未给定,时默认为-NoValue-
% o(不是0),表示标准可选.使用[]给出
% {},使用{}给出

% 一个复杂的命令
\NewDocumentCommand{\foo}{mg}{
    \IfNoValueTF{#2}%   翻译:如果#2是Novalue
        {One parameter: #1{}.}
        {Two parameter3: #1{}, #2{}.}
}

% 另一个复杂的命令
\NewDocumentCommand{\foob}{oom}{%
    Text with #3 and perhaps \textcolor{blue}{#1} and \textcolor{red}{#2}%
}

% 自定义一个复杂的环境:不能够使用o ---> 原因:
\NewDocumentEnvironment{Test}{O{dft1}O{dft2}m}%
    {%
        \ifthenelse{\isempty{#1}}
        {%
            \ttfamily
            \noindent This is \#1:~ dft1\\
            This is \#2:~ #2\\
            This is \#3:~ #3 
        }
        {%
            \ttfamily
            \noindent This is \#1:~ #1\\
            This is \#2:~ #2\\
            This is \#3:~ #3
        }
    }{}

% 使用xifthen做判断
\newcommand{\optarg}[1][]{%
    \ifthenelse{\isempty{#1}}%
    {}% if #1 is empty
    {(((#1)))}% if #1 is not empty
}

% 使用{}进行复合赋值(分割)
\newcommandx\gougu[3][1={a=2},2=b,3=c,usedefault]
    {#1^2 + #2^2 = #3^2}


\title{自定义宏包}
\author{\calligra Eurekax}
\date{\today}
\begin{document}
\maketitle
\section{自定义宏}

\subsection{NewDocumentCommand}

\begin{formal}[blue!5]{blue}
    \baa{Mouse}\\
    \bab{Fish}
\end{formal}

\begin{formal}[blue!5]{blue}
    \foo{hello}\\
    \foo{Hello}{world}
\end{formal}

\begin{formal}[blue!5]{blue}
    \foob{第一个必选参数}\\
    \foob[第一个可选参数]{第一个必选参数}\\
    \foob[第一个可选参数][第二个可选参数]{第一个必选参数}
\end{formal}



\subsection{NewDocumentEnvironment}
\begin{formal}{green}
    \begin{Test}{第一个必选参数}
    \end{Test}
\end{formal}

\begin{formal}{green}
    \begin{Test}[第一个可选参数]{第一个必选参数}
    \end{Test}
\end{formal}

\begin{formal}{green}
    % 使用*来占位
    \begin{Test}[][第二个可选参数]{第一个必选参数}
    \end{Test}
\end{formal}

\begin{formal}{green}
    \begin{Test}[第一个可选参数][第二个可选参数]{第一个必选参数}
    \end{Test}
\end{formal}

\subsection{新方式宏参数指定}
\begin{formal}[yellow!5]{yellow}
    \begin{center}
        \ttfamily
        \begin{tabular}{l|llllp{8em}}
            参数指定 & 输入值 & \#1 & \#2 & \#3\\
            \hline\\
            m~m & \{foo\}\{bar\} & foo & bar\\
            o~m & \{foo\} & -Novalue & foo\\
            o~o~m & [foo]\{bar\} & foo & -Novalue- & bar\\
            o~o~m & [foo][bar]{Need} & foo & bar & Need\\
            O\{dft1\}~O\{dft2\}~m & \{foo\} & dft1 & dft2 & foo\\
            m~O\{default\} & \{foo\} & foo & default\\
            m~O\{default\} & \{foo\}[bar] & foo & bar\\
            m~O\{default\} & [bar] & 报错
        \end{tabular}
    \end{center}
\end{formal}

\section{宏定义进阶}
\subsection{(类似)关键字参数:xargs}
    \begin{align}
        \gougu\\ 
        \gougu[3][4][5]\\ 
        \gougu[][d]\\ 
        \gougu[][][g] 
    \end{align}

\subsection{条件判断:xifthen}
Testing \verb|\optarg|: \optarg% prints nothing

Testing \verb|\optarg[]|: \optarg[]% prints nothing

Testing \verb|\optarg[test]|: \optarg[test]% prints (((test)))

\subsection{宏包可选参数}
\begin{formal}{red}
    Option选择命令的使用: \verb |\usepackage[test]{Article}|\\
    \testcmd{}
\end{formal}

\section{重写绘图宏包}
\subsection{重写plott命令}
\begin{formal}{blue}
    原始代码:
\begin{verbatim}
\begin{Plot}{scale=0.8}{>=stealth}
    \plott{blue!40, thick}{-1:4}{\x*\x-2*\x-1}
    \plott{red, thick}{0:pi}{3*sin(\x r)}
\end{Plot}
\end{verbatim}

    \noindent 运行结果:
    \begin{center}
        \begin{Plot}{scale=0.8}{>=stealth}
            \plott{blue!40, thick}{-3:4}{0.3*\x+1}
            \plott{red, thick, samples=200}{0:pi}{3*sin(4*\x r)}
        \end{Plot}
    \end{center}
\end{formal}

\begin{formal}{red}
    简化后代码:
\begin{verbatim}
\plot{exp(-\x)*sin(2*\x)+ln(\x)*cos(\x)}[0:4]    
\end{verbatim}  

    \noindent 运行结果:
    \begin{center}
        \begin{TikAxis}{scale=0.5}
            \plot[red, thick]{exp(-\x)*sin(2*\x)+ln(\x)*cos(\x)}[0.2:5]
        \end{TikAxis}
    \end{center}
\end{formal}
\newpage

\subsection{重写polarplott}
\begin{formal}{blue}
    原始代码:\\
    \verb |\polarplott{scale=2}{red, domain=0:720}{0.01/pi*\t}|

    \noindent 运行结果:
    \begin{center}
        \polarplott{scale=2}{red, domain=0:720}{0.01/pi*\t}             
    \end{center}
\end{formal}

\begin{formal}{red}
简化后代码:
\begin{verbatim}
\polarplot{sin(2*\t)}
\polarplot[scale=2]{sin(2*\t)}
\polarplot[][blue, thick, domain=0:180]{sin(2*\t)}
\end{verbatim}
运行结果:
    \begin{center}
        \polarplot{sin(2*\t)}
        \polarplot[scale=2]{sin(2*\t)}
        \polarplot[][blue, thick, domain=0:180]{sin(2*\t)}
    \end{center}
\end{formal}

\newpage
\subsection{重写paraplott(绘制二维参数方程)}
\begin{formal}{blue}
    原始代码:
\begin{verbatim}
\paraplott{scale=1}{{10*cos(t)}, {sin(t)}}{thick, color=blue}{}    
\end{verbatim}

    \noindent 运行结果:
    \begin{center}
        \paraplott{scale=1}{{3*cos(t)}, {sin(t)}}{thick, color=blue}{}   
    \end{center}
\end{formal}
\begin{formal}{red}
    简化后代码:
\begin{verbatim}
\paraplot{{3*cos(t)}, {sin(t)}}
\paraplot[scale=0.3]{{3*cos(t)}, {sin(t)}}
\paraplot[scale=0.3]{{3*cos(t)}, {sin(t)}}[0:1.5*pi]
\end{verbatim}
    \noindent 运行结果:
    \begin{center}
        \paraplot{{3*cos(t)}, {sin(t)}}
        \paraplot[scale=0.3]{{3*cos(t)}, {sin(t)}}
        \paraplot[scale=0.4]{{3*cos(t)}, {sin(t)}}[0:1.5*pi]
    \end{center}
\end{formal}

\newpage
\subsection{重写paraplotzz(绘制三维参数方程)}

\begin{formal}{blue}
    原始代码:
\begin{verbatim}
\paraplotzz{scale=0.5} 
    {{sin(deg(t))},{cos(deg(t))},{t}} 
    {}
    {view = {60}{90},axis lines = center}
\end{verbatim}

    \noindent 绘制结果
    \begin{center}
        \paraplotzz{scale=0.5} 
            {{sin(deg(t))},{cos(deg(t))},{t}}
            {thick, red}
            {view = {60}{30},axis lines = center}  
    \end{center}
\end{formal}

\begin{formal}{red}
    简化后代码: 
\begin{verbatim}
\paraplotz{{sin(deg(t))}, {cos(deg(t))}}
\paraplotz[scale=0.3]{{sin(deg(t))}, {cos(deg(t))}}
\paraplotz[scale=0.5]{{sin(deg(t))}, {cos(deg(t))}}[-2*pi:2*pi]
\end{verbatim}

    \noindent 运行结果:
    \begin{center}
        \paraplotz{{sin(deg(t))}, {cos(deg(t))}}
        \paraplotz[scale=0.3]{{sin(deg(t))}, {cos(deg(t))}}
        \paraplotz[scale=0.5]{{sin(deg(t))}, {cos(deg(t))}}[-2*pi:2*pi]
    \end{center}
\end{formal}

\newpage
\subsection{重写plotzz(三维绘图)}

\begin{formal}{blue}
    原始代码:
\begin{verbatim}
\plotzz{}{sin(deg(sqrt(x^2+y^2)))/sqrt(x^2+y^2)}{surf}{}{图例}
\end{verbatim}

    \noindent 运行结果:
    \begin{center}
        \plotzz{}{sin(deg(sqrt(x^2+y^2)))/sqrt(x^2+y^2)}{surf}{}{图例}
    \end{center}
\end{formal}

\begin{formal}{red}
    简化后代码:
\begin{verbatim}
\plotz[scale=1.5]{sin(deg(sqrt(x^2+y^2)))/sqrt(x^2+y^2)}[mesh]   
\end{verbatim}

    \noindent 运行结果:
    \begin{center}
        \plotz[scale=1.5]{sin(deg(sqrt(x^2+y^2)))/sqrt(x^2+y^2)}[mesh]
    \end{center}
\end{formal}


\end{document}