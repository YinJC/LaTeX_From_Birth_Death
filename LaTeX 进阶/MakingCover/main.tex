% 生成最终的图象时把第一个文档类取消注释即可
%\documentclass[10pt,varwidth]{standalone}
\documentclass[12pt]{article}
% 1.必须添加varwidth选项,不然就会报错
\PassOptionsToPackage{quiet}{fontspec}
\usepackage{ctex}
\usepackage{geometry}
% 必须要保证绘图的纸张足够的大
\geometry{a2paper,left=1in,right=1in,top=1in,bottom=1in}
\usepackage{xifthen}
\usepackage{xfp}
\usepackage{xcolor}
\usepackage{pgfplots}
\usepackage{pgfplotstable}
\pgfplotsset{compat=1.16}
% 2.引用的tikz库
\usetikzlibrary {matrix, chains, trees, decorations}
\usetikzlibrary {arrows.meta, automata,positioning}
\usetikzlibrary {decorations.pathmorphing, calc}
\usetikzlibrary {backgrounds, mindmap,shadows}
\usetikzlibrary {patterns, quotes, 3d, shadows}
\usetikzlibrary {graphs, fadings, scopes}
\usetikzlibrary {arrows, shapes.geometric}
\usepgflibrary {shadings}

\tikzset{
    >={Latex[length=6mm, width=2mm]}
}


\newcommand{\scale}[2]{%
    \scalebox{#1}[#1]{#2}}
% \usepackage[lite,subscriptcorrection,slantedGreek,nofontinfo]{mtpro2}



\begin{document}

\leavevmode\kern-30em{   
    \textcolor{gray}{
    \rotatebox{12}{
        \scale{60}{$\displaystyle\int$}
    }}
}
%
%
% \parbox{.5em}{AMS}
% \vbox{}
% \vbox{
%   First line of text \\
%   Second line of text \\
%   Third line of text
% }
% \vbox{
%   \hbox to 5cm{First line of text}
%   \hbox to 8cm{Second line of text}
%   \hbox to 6cm{Third line of text}
% }
Hello \leavevmode World

\ifhmode
  Currently in horizontal mode.
\else\ifvmode
  Currently in vertical mode.
\fi\fi
\ifhmode
  Currently in horizontal mode.
\else\ifvmode
  Currently in vertical mode.
\fi\fi

\end{document}